\documentclass[]{scrartcl}
\title{Vorlesung Analysis II}
\usepackage{amsmath,amssymb,amsfonts}
\usepackage{stmaryrd}
\usepackage{mathtools}
\usepackage{latexsym}
\usepackage{graphicx}
\usepackage{tikz}
\usepackage{xcolor}
\usepackage[most]{tcolorbox}
\usepackage{soul}
\usepackage{ upgreek }
\usepackage{hyperref}
\usepackage{tipa}
\usepackage[dvipsnames]{xcolor}
\hypersetup{
	colorlinks=true,
	linkcolor=blue,
	filecolor=magenta,      
	urlcolor=cyan,
	pdftitle={Overleaf Example},
	pdfpagemode=FullScreen,
}
\newcommand{\redcircle}[1]{%
	\tikz[baseline=(char.base)]{
		\node[shape=circle, draw=red, text=red, thick, inner sep=1pt] (char) 
		{\textbf{#1}};
	}%
}
\newcommand{\bluecircle}[1]{%
	\tikz[baseline=(char.base)]{
		\node[shape=circle, draw=blue, text=blue, thick, inner sep=1pt] (char) 
		{\textbf{#1}};
	}%
}
\newcommand{\blackcircle}[1]{%
	\tikz[baseline=(char.base)]{
		\node[shape=circle, draw=black, text=black, thick, inner sep=1pt] 
		(char) 
		{\textbf{#1}};
	}%
}
\newcommand{\orangecircle}[1]{%
	\tikz[baseline=(char.base)]{
		\node[shape=circle, draw=orange, text=orange, thick, inner sep=1pt] 
		(char) 
		{\textbf{#1}};
	}%
}
\newcommand{\redul}[1]{\setulcolor{red}{\ul{#1}}}
\newcommand{\blueul}[1]{\setulcolor{blue}{\ul{#1}}}
\newcommand{\yelul}[1]{\setulcolor{yellow}{\ul{#1}}}
\newcommand{\greenul}[1]{\setulcolor{green}{\ul{#1}}}
\newcommand{\oraul}[1]{\setulcolor{orange}{\ul{#1}}}
\setul{1pt}{3pt} % Linienhöhe und Abstand zum Text (optional anpassbar)

\setlength{\topmargin}{-.5in} \setlength{\textheight}{9.25in}
\setlength{\oddsidemargin}{0in} \setlength{\textwidth}{6.8in}
\setlength{\parindent}{0pt}

\begin{document}
	\maketitle
	\underline{\textbf{Teil 3: Gewöhnliche Differentialgleichungen}}\\
	\\
	\\textbf
	\underline{an 19: Bernoullische und Euler-homogene DGL}\\
	\\
	\textbf{\underline{\underline{Stichworte:} Bernoullische DGL, (Euler-)homogene DGL}}\\
	\\
	\textbf{\underline{Literatur:}} \blueul{[Hoffmann], Kapitel 7.4/5}\\
	\\
	\textbf{19.1. \underline{Einleitung:}} Die Bernoullische DGL ist eine spezielle Version der Linearen DGL 1. Ordnung und wird darauf zurückgeführt. Die Euler-Homogene DGL ist ein DGL 1. Ordnung mit einem Term abhängig von $\frac{y}{x}$ auf der rechten Seite und kann auf eine DGL mit getrennten Variablen zurückgeführt werden.\\
	\\
	\textbf{19.2. \underline{Def.:}} Die DGL \fcolorbox{red}{white}{$y'=f(x)y+g(x)y^\alpha$} \blackcircle{*}, $\alpha\in\mathbb{R}\backslash\{0,1\}$,\\
	heißt \redul{Bernoullische DGL}, wobeie f,g: j $\rightarrow \mathbb{R}$ stetig, j ein IV.\\
	(Für $\alpha$=1 ist dies eine homogene Lineare DGL 1. Ordnung, für $\alpha=0$ die (inhomogene) Lineare DGL 1. Ordnung.)\\
	\\
	\textbf{19.3. \underline{Vereinbarung:}} Betrachten nur Lösungen y:$j_0\rightarrow\mathbb{R}, j_0\subseteq j$ Teil IV mit \redul{$y(x)\textgreater0$} für $x\in j_0$\\
	(für spezielle $\alpha$ ginge es, auch y(x)$\leq$0 zuzulassen. Sofern $0^\alpha=0$ definiert ist, ist auch y = 0 Lösung.)\\
	\\
	\textbf{19.4. \underline{Satz:}} Die Transformation \redul{$u(x)=y(x)^{1-\alpha}$}, $x\in j_0$, liefert: \greenul{y Lsg. von $\blackcircle{*}$}$\Leftrightarrow$ \greenul{u löst auf $j_0$ die Lineare DGL}\\
	\fcolorbox{red}{white}{$u'=(1-\alpha)f(x)u+(1-\alpha)g(x)$} $\blackcircle{+}, u(x)\textgreater0, x\in j_0.$\\
	\\
	\textbf{19.5. \underline{Bew.:}} "$\Rightarrow$": Ist y Lsg. von $\blackcircle{*}$, dann gilt für u:\\
	\begin{align}
		'=(1-\alpha)y^{-\alpha}y'&=(1-\alpha)y^{-\alpha}[f(x)y+g(x)y^\alpha]\\
		&=(1-\alpha)f(x)u+(1-\alpha)g(x).
	\end{align}\\
	"$\Leftarrow$": Ist u Lsg. von $\blackcircle{+}$, folgt\\
	\begin{align}
		y'=\frac{1}{1-\alpha}u^{\frac{1}{1-\alpha}-1}u'&=\frac{1}{1-\alpha}u^{\frac{\alpha}{1-\alpha}}[(1-\alpha)f(x)u+(1-\alpha)g(x)]\\
		&=f(x)u^{\frac{1}{1-\alpha}}+g(x)u^{\frac{\alpha}{1-\alpha}}=f(x)y+g(x)x^\alpha.
	\end{align}\\
	\strut\hfill$\square$\\
	\textbf{19.6. \underline{Bsp.:}} \redul{$y'=xy-3xy^2, y(0)=\frac{1}{4}$}.\\
	Mit $\alpha=2$, $u(x)=y(x)^{-1}=\frac{1}{y(x)}$ in einem IV $j_0\subseteq \mathbb{R}, 0\in j_0,\\
	y(x)\textgreater0$ für $x \in j_0$ transformieren wir die DGL um in\\
	\greenul{$u'=-xu+3x, u(0)=4$}. \textopencorner $u'=(\frac{1}{y}'=-\frac{xy}{y^2}+3x=-xu+3x)$\textcorner\\
	Nach \blueul{Bsp. 18.8.} ist $u(x)=b\exp (-\frac{1}{2}x^2)+3$ die aöög. Lsg. dieser DGL, die Lösung der ursprünglichen DGL für y ist dann \greenul{y(x)=$\frac{1}{u(x)}=\frac{1}{3+\exp(-x^2/2)}$}.\\
	\\
	\textbf{19.7. \underline{Def.:}} Die DGL \fcolorbox{red}{white}{$y'=f(\frac{y}{x})$} \blackcircle{*}\\
	wird meist als \redul{"homogene" DGL} bezeichnet.\\
	Um Verwechslungen mit Homogenität bei linearen DGLn auszuschließen, nennen wir sie \redul{Euler-homogene DGL}.\\
	\\
	\textbf{19.8. \underline{Verfahren:}} Die Substitution \redul{$u=\frac{y}{x}$} für $x\neq0$ bzw. \redul{y=xu} liefert \redul{$f(u)=u+xu'$}, also \redul{$u'=\frac{f(u)-u}{x}$}, eine DGL mit \redul{"getrennten Variablen"}.\\
	\\
	\textbf{19.9. \underline{Bsp.:}} \redul{$y'=\frac{x-y}{x}, y(2)=3$}.\\
	Die r.s. ist $=1-\frac{y}{x}$, mit \redul{$f(u):=1-u$} ergibt die Substitution \redul{y=xu} die AWA $u'=\frac{1-2u}{x}, u(2)=\frac{3}{2}.$\\
	Für $x\textgreater0, u\textgreater\frac{1}{2}$, berechnen wir nach \blueul{17.5.} \textopencorner dort mit $f(x)=\frac{1}{x}, g(u)=\frac{1}{1-2u}$\textcorner:\\
	$\int_{3/2}^{u(x)}\underbrace{\frac{1}{1-2x}}_{0}ds\overset{!}{=}\int_{2}^{x}\frac{1}{t} dt$, folglich $-\frac{1}{2}\ln(2s-1)\underset{3/2}{\overset{u(x)}{\rule{0.4pt}{0.5cm}}} \overset{!}{=}\ln(x)-\ln(2)$\\
	 bzw. $\ln(2u(x)-1)-\ln(2)\overset{!}{=}2(\ln(2)-ln(x)).$\\
	 Also ist $2u(x)-1=e^{3\ln(2)-2\ln(x)}=2^3x^{-2},$ also $2u(x)=\frac{8}{x^2}+1$,\\
	 dies führt zu $u(x)=\frac{4}{x^2}+\frac{1}{2}$ und schließlich zu \greenul{$y(x)=xu(x)=\frac{4}{x}+\frac{x}{2}=\frac{8+x^2}{2x}$}.\\
	 \textopencorner Probe: $y'=-\frac{4}{x^2}+\frac{1}{2}\overset{!}{=}1-\frac{1}{x}\underbrace{(\frac{4}{x}+\frac{x}{2})}_{y}\checkmark, y(2)=\frac{8+4}{4}=3\checkmark$\textcorner
\end{document}