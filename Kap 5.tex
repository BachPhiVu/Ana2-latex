\documentclass[]{scrartcl}
\title{Vorlesung Analysis II}
\usepackage{amsmath,amssymb,amsfonts}
\usepackage{stmaryrd}
\usepackage{mathtools}
\usepackage{latexsym}
\usepackage{graphicx}
\usepackage{tikz}
\usepackage{xcolor}
\usepackage{soul}
\usepackage{ upgreek }
\usepackage{hyperref}
\usepackage{tipa}
\usepackage[dvipsnames]{xcolor}
\hypersetup{
	colorlinks=true,
	linkcolor=blue,
	filecolor=magenta,      
	urlcolor=cyan,
	pdftitle={Overleaf Example},
	pdfpagemode=FullScreen,
}
\newcommand{\redcircle}[1]{%
	\tikz[baseline=(char.base)]{
		\node[shape=circle, draw=red, text=red, thick, inner sep=1pt] (char) 
		{\textbf{#1}};
	}%
}
\newcommand{\bluecircle}[1]{%
	\tikz[baseline=(char.base)]{
		\node[shape=circle, draw=blue, text=blue, thick, inner sep=1pt] (char) 
		{\textbf{#1}};
	}%
}
\newcommand{\blackcircle}[1]{%
	\tikz[baseline=(char.base)]{
		\node[shape=circle, draw=black, text=black, thick, inner sep=1pt] (char) 
		{\textbf{#1}};
	}%
}
\setul{1pt}{3pt} % Linienhöhe und Abstand zum Text (optional anpassbar)

\setlength{\topmargin}{-.5in} \setlength{\textheight}{9.25in}
\setlength{\oddsidemargin}{0in} \setlength{\textwidth}{6.8in}
\setlength{\parindent}{0pt}

\begin{document}
	\textbf{\underline{Teil 1: Differentialrechnung im $\mathbb{R}^n$}}\\
	\\
	\textbf{\underline{an5: Partielle und totale Ableitungen}}\\
	\\
	\textbf{\underline{\underline{Stichworte:} Funktionalmatrix, Gradient, 
	Kettenregel, Richtungsableitungen}}\\
	\\
	\textbf{\underline{Literatur:}}\setulcolor{blue}\ul{[Hoff], Kapitel 9.4}\\
	\\
	\textbf{5.1. \underline{Einleitung:}} Die totale Ableitung liefert einen 
	einfachen Weg, Richtungsableitungen zu berechnen. Wirdefinieren für m=1 den 
	Gradienten und beweisen die allgemeine Kettenregel.\\
	\\
	\textbf{5.2. \underline{Konvention:}} Betrachten Funktionen 
	$f:U\rightarrow\mathbb{R}^m$ mit $U\subseteq \mathbb{R}^n$.\\
	Sei $a\in U$ ein innerer Punkt von U, d.h. $\exists s\textgreater0: 
	U_a^s\subseteq U.$\\
	Wir bezeichnen die Menge aller Richtungsvektoren $v\in \mathbb{R}^n$ mit 
	\setulcolor{yellow}\ul{$S^{n-1}$}:=$\{v\in \mathbb{R}^n; ||v||_2=1\}$, die 
	(n-1)-dimensionale \setulcolor{red}\ul{Sphäre im $\mathbb{R}^n$}.\\
	$\bullet$ Für einen inneren Punkt $a\in U$ und ein $v\in S^{n-1}$ haben wir 
	in \setulcolor{blue}\ul{an4.8}\\
	$D_vf(a):=\lim\limits_{0\neq t\rightarrow0}\frac{1}{t}(f(a+tv)-t(a))$\\
	als Richtungsableitung von f in a in Richtung v definiert.\\
	(Diese Def. benutzt, dass $a+t v\in U$ ist aööe $t\in\mathbb{R}$ mit 
	hinreichend kleinen $|t|$)\\
	$\bullet$\underline{Speziell:} Ist $v=e_i$ der i-te Einheitsvektor, so ist\\
	$D_if(a)=\frac{\delta f}{\delta \xi_i}(a)=f_{\xi_i}(a):=(D_{e_i}f)(a)$ die 
	i-te partielle Ableitung.\\
	\\
	\textbf{5.3. \underline{Satz:}} \underline{Vor.:} \setulcolor{green}\ul{f 
	in a diff'bar} (d.h. total diff'bar), \ul{$v\in S^{n-1}$}.\\
	\underline{Beh.:} f ist \ul{in Richtung v in a diff'bar} und 
	\ul{($D_vf$)(a)}=\ul{$\underbrace{(Df)(a)}_{\in\mathbb{R}^{m x n}}\cdot 
	\underbrace{v}_{\in\mathbb{R}^n}$}=$\underbrace{f'(a)(v)}_{\text{mit} 
	f'(a)\in Hom(\mathbb{R}^n,\mathbb{R}^m)\text{ausgedrückt, dh als lineare 
	Abb.}}$.\\
	\underline{Bew.:}$\frac{1}{t}(f(a+tv)-f(a))=\frac{1}{t}(f'(a)(tv)+o(||tv||_2)
	=\underline{Bew.:}\frac{1}{t}(f(a+tv)-f(a))=\frac{1}{t}(f'(a)(tv)+o(||tv||_2))=
	 f'(a)(v)+\underbrace{\underbrace{\frac{|t|\cdot||v||_2}{t}}_{\text{beschränkt}}
	  \cdot 
	 \underbrace{\frac{o(||tv||_2)}{||tv||_2}}_{\xrightarrow{t\rightarrow0}0}}_{\xrightarrow{t\rightarrow0}0}
	$.\\
	\strut\hfill$\square$\\
	\textbf{5.4 \underline{Bsp.:}} $f:\mathbb{R}^2\rightarrow\mathbb{R}^3, f(x,y)=\begin{pmatrix}
		x^2-y\\3x\\sin(y)
	\end{pmatrix}$ gibt $f'(x,y)=\begin{pmatrix}
		2x&-1\\3&0\\0&cos(y)
	\end{pmatrix},$ sei $a=\begin{pmatrix}
		1\\\pi/4
	\end{pmatrix}\in \mathbb{R}^2.$\\
	Dann: $f'(\begin{pmatrix}
		1\\\pi/4
	\end{pmatrix})=\begin{pmatrix}
		2&-1\\3&0\\0&\frac{1}{2}\sqrt{2}
	\end{pmatrix}$ und sei $v=\frac{1}{5}\begin{pmatrix}
		3\\4
	\end{pmatrix}\in\mathbb{R}^2$, haben $||v||_2=\frac{1}{5}\cdot \sqrt{3^2+4^2}=1.$\\
	sich dann als $f'\begin{pmatrix}
		1\\\pi/4
	\end{pmatrix}\cdot \frac{1}{5}(\begin{pmatrix}
		\textcolor{magenta}{3}\\\textcolor{magenta}{4}
	\end{pmatrix})=\frac{1}{5}\begin{pmatrix}
		2\cdot \textcolor{magenta}{3}+(-1)\cdot \textcolor{magenta}{4}\\
		3\cdot \textcolor{magenta}{3}+0\cdot \textcolor{magenta}{4}\\
		0\cdot \textcolor{magenta}{3}+\frac{1}{2}\sqrt{2}\cdot \textcolor{magenta}{4}
	\end{pmatrix}=\frac{1}{5}\begin{pmatrix}
		2\\9\\2\sqrt{2}
	\end{pmatrix}.$\\
	Die partiellen Ableitungen sind $D_1 f(a)=f'\begin{pmatrix}
		1\\\pi/4
	\end{pmatrix}\cdot e_1=\begin{pmatrix}
		2\\3\\0
	\end{pmatrix}, D_2f(a)=f'\begin{pmatrix}
		1\\\pi/4
	\end{pmatrix}\cdot e_2 =\begin{pmatrix}
		-1\\0\\\frac{1}{2}\sqrt{2}
	\end{pmatrix}.$\\
	\\
	\textbf{5.5 \underline{Folgerungen:}} Sei f in a Diff'bar.\\
	\underline{Beh.:}\\
	(a) f ist in a in jeder Koordinate partiell diff'bar,\\
	(b)\setulcolor{green}\ul{$(D_if)(a)=f'(a)\cdot e_i$}, und dies ist die i-te spalte von f'(a), denn Sie wissen ja:\\
	\textcolor{blue}{die Spalten einer Matrix sind genau die Bildere der Einheitsvektoren.}\\
	Also sind die Spalten von f' genau die Pariellen Ableitungen von f.\\
	(c) Es ist $f'(a)=Df(a)=\begin{pmatrix}
		(D_1f_1)(a)& \cdots & (D_nf_1)(a)\\
		(d_1f_2)(a)& \cdots & (D_nf_2)(a)\\
		\vdots& & \vdots\\
		(D_1f_m)(a)&\cdots & (D_nf_m)(a)
	\end{pmatrix}$\\
	(d) Es ist \ul{$(D_jf_j)(a)=pr_i(D_jf(a))$}, also $D_jf_i=pr_i\circ D_jf$ für alle $j\in \{1,...,n\}, i\in \{1,...,m\}$.\\
	\underline{Bew.:} (a),(b),(c): Klar mit \setulcolor{blue}\ul{5.3} und $v=e_i$.\\
	Für (d): Haben $f(a)=\begin{pmatrix}
		f_1(a)\\\vdots\\f_m(a)
	\end{pmatrix}$ und $d_jf(a)=\begin{pmatrix}
		D_jf_1(a)\\D_jf_2(a)\\\vdots\\D_jf_m(a)
	\end{pmatrix},$\\
	also $pr_j(D_jf_i(a))_{1\leq i\leq m|1\leq j\leq n}\in\mathbb{R}^{m x n}$\\
	die \setulcolor{red}\ul{Jacobimatrix/Funktionalmatrix} von f in a.\\
	\underline{Bem.:} "$\Leftarrow$" Kann in \setulcolor{blue}\ul{5.5} nicht gelten, die Existenz der Partiellen Ableitungen reicht nicht zum Nachweis der Differenzierbarkeit! (vgl. \ul{4.15, 4.16})\\
	\\
	\textbf{5.7 \underline{Fall:}} sei \underline{\underline{m=1}}, also f ein \underline{Skalarfeld}.\\
	\setulcolor{red}
	Dann: \begin{align}
		f'(a)=(d_1f(a),...,D_nf(a))&=:(grad f(a))^T &"\ul{\textbf{Gradient}}"\\
		&=:(\triangledown f(a))^T &"\ul{\textbf{Nabla}}"
	\end{align}\\
	Wir nennen den Spaltenvektor \setulcolor{yellow} \ul{grad f(a)}=$\begin{pmatrix}
		D_1f(a)\\\vdots\\D_nf(a)
	\end{pmatrix} \in \mathbb{R}^n$ den \setulcolor{red}\ul{Gradient von f in a}.\\
	\setulcolor{yellow} mit \ul{$\triangledown$}:=$\begin{pmatrix}
		D_1\\\vdots\\D_n
	\end{pmatrix}$ bezeichnen wir den \setulcolor{red}\ul{Nabla-Operator}.\\
	\\
	Somit:\begin{align}
		&f(x)=f(a)+(grad f(a))^T\cdot(x-a)+o(||x-a||)\\
		\Rightarrow &f(x)=f(a)+\textless grad f(a), x-a\textgreater + o(||x-a||).
	\end{align}\\
	Sei grad f(a)$\neq o$, und betrachte alle $v\in S^{n-1}$.\\
	Dann gilt: \setulcolor{green}\ul{$|D_vf(a)|$}=$|f'(a)(v)|=|(grad f(a))^T\cdot v|\\
	=|\textless grad f(a),v\textgreater|\leq ||grad f(a)||\cdot ||v||_2$ (cauchyscchwarzungleichung Anhang 7 in an1)\\
	Wobei "=" genau dann gilt, wenn v parallel zu grard f(a), d.h. ex. $t\in\mathbb{R}\backslash\{0\}: grad f(a)=tv,$\\
	in diesem Fall wird für $D_cf(a)$ der maximale Wert angenommen.\\
	Für $\tilde{v}:=\frac{1}{||grad f(a)||_2} grad f(a)$ gilt demnach:\\
	$|D_{\tilde{v}}f(a)|=||grad f(a)||_2.$\\
	\\
	\textbf{5.8. \underline{FAZIT:}} \setulcolor{green}\ul{grad f(a) ist die RIchtung maximaler Steigung von f in a}\\
	(welche dann $||grad f(a)||_2$ beträgt).\\
	\\
	\textbf{5.9. \underline{Veranschaulichung:}} Der Graph von f, nähmlich G(f):$\{\begin{pmatrix}
		x\\f(x)
	\end{pmatrix}\in\mathbb{R}^{n+1}\}$ (falls $U=\mathbb{R}^n$),\\
	wird in $\begin{pmatrix}
		a\\f(a)
	\end{pmatrix}$ approximiert durch\\
	$\xi_{n+1}= f(a)+\textless grad f(a), x-a\textgreater,$\\
	und dies ist die Glg. für eine n-dim. Hyperebene im $\mathbb{R}^{n+1}$!\\
	Diese heißt \setulcolor{red}\ul{Tangentialhyperebene von f im Punkt a}.\\
	\\
	\textbf{5.10 \underline{Bsp. mit n=2:}} $f(x,y)=x^2-3y, a=\begin{pmatrix}
		\alpha_1\\\alpha_2
	\end{pmatrix}	\rightarrow grad f(a)=(2\alpha_1,-3\alpha_2)^T,$ und\\
	$\xi_3=\alpha_1^2-3\alpha_2+2\alpha_1(\xi_1-\alpha_1)-3\alpha_2(\xi_2-\alpha_2)$ ist die Glg. der Tangentialhyperebene im $\mathbb{R}^3.$\\
	\\
	\textbf{5.11. \underline{Fall:}} Sei \underline{\underline{m=n}}, also f ein \underline{Vektorfeld}.\\
	Dann heißt \setulcolor{yellow}\ul{$div f(a)=\textless \triangledown, f\textgreater(a)$}:=$\sum_{i=1}^{n}D_if_i(a)=spur f'(a)\in\mathbb{R}\leftarrow$ [Erinnerung Lineare Algebra: $spur A = \sum_{i=1}^{n}\alpha_{ij}$, wenn $A=(\alpha_{ij})\in \mathbb{R}^{m x n}$, heißt \setulcolor{red}\ul{Spur} der Matrix A.]\\
	die \ul{Divergenz von f in a.}\\
	Die Funktionalmatrix ist quadratisch: $Df(a)=f'(a)=\begin{pmatrix}
		D_1f_1(a)&D_2f_1(a)&\cdots&D_nf_1(a)\\
		\vdots&\vdots&\vdots&\vdots\\
		D_1f_n&D_2f_n&\cdots&d_nf_n(a)
	\end{pmatrix}\in \mathbb{R}^{n x m}.$\\
	Ihre Determinante heißt \ul{Funktionaldeterminante} bzw. \ul{Jacobideterminante}.\\
	\\
	Eine wichtige Rechenregel für das Ableiten verketteter Funktionen im Mehrdimensionalen ist die (allgemeine)\\
	\textbf{5.12. \ul{Kettenregel:}} \underline{Vor.:} $ U \subseteq\mathbb{R}^n; U_1\subseteq\mathbb{R}^m; a\in U\xrightarrow{f} U_1\xrightarrow{g}\mathbb{R}^k$, \setulcolor{green}\ul{f in a diff'bar, g in f(a) diff'bar}\\
	(insb. a innerer Punkt von U, f(A) innerer Punkt von $U_1$).\\
	\underline{Beh.:}\ul{$g\circ f$ in a diffbar},  $\underbrace{D(g\circ f) (a)}_{\in\mathbb{R}^{k x n}}=\underbrace{(Dg)(f(a))}_{\in\mathbb{R}^{k x m}}\cdot \underbrace{(Df)(a)}_{\in\mathbb{R}^{m x n}}$.\\
	\underline{Bew.:} Setze B:= f(a).\\
	dann gilt für $r_1(x)=o(||x-a||),$ dass $ f(x)=f(a)+f'(a)(x-a)+r_1(x), \blackcircle{*}$\\
	und für $r_2(y)=o(||y-b||),$ dass $ g(y)=g(b)+g'(b)(y-b)+r_2(y).$\\
	$\xRightarrow{y=f(x)|b=f(a)} g(f(x))=g(f(a))+g'(f(a))(f(x)-f(a))+r_2(f(x))\\
	\xRightarrow{\blackcircle{*}}g \circ f (a)+\underbrace{(Df)(f(a))\cdot (Df)(a\cdot(x-a))}_{=D(g\circ f)(a)\rightarrow\text{Beh.}}+g'(f(a))(r_1(x))+r_2(f(x))$\\
	\\
	Noch z.z.: $g'(f(a))(r_1(x))+r_2(f(x))=o(||x-a||).$\\
	(Def. für eine Matrix $A\in \mathbb{R}^{m x n}$ den wert 
	$||A||_\infty:\max_{i,j}|a_{ij}|$, wenn $A=(a_{ij})_{i,j}.)$\\
	$\bullet$ Es gilt:\\
	$||\underbrace{g'(f(a))}_{\in\mathbb{R}^{k x m}}\cdot 
	\underbrace{r_1(x)}_{\in \mathbb{R}^m}||_\infty \leq\redcircle{Ü} 
	m\underbrace{||g'(f(a))||_\infty}_{\text{maximaler Eintrag der Matrix 
	g'(f(a)) im Betrag, unabhängig von x}}\cdot ||r_1(x)||_\infty=o(||x-a||)$ 
	nach Vor. an $r_1$.\\
	$\bullet$ Bleibt, z.z.: $r_2(f(x))=o(||x-a||).$\\
	Haben $r_2(y)=o(||y-b||),$ d.h. 
	$\frac{r_2(y)}{||y-b||_\infty}\xrightarrow{y\rightarrow b}o.$\\
	\\
	Wähle $\mu \textgreater 0$. Dann ist für y nahe b: $r_2(y)\textless 
	\mu\cdot||y-b||_\infty.$\\
	Es folgt: \begin{align}
		r_2(f(x))&\textless 
		\mu||f(x)-f(a)||_\infty=\mu||f'(a)(x-a)+r_1(x)||_\infty\\
		&\leq\mu \underbrace{\underbrace{||f'(a)||_\infty}_{\text{Konstant d.h. 
		m abh. von x}}\cdot||x-a||_\infty}_{\leq 
		\tilde{\mu}||x-a||\infty}+\underbrace{\mu||r_1(x)||\infty}_{=o(||x-a||)},
	\end{align}
	also $\frac{r_2 (f(x))}{||x-a||_\infty}\xrightarrow{x\rightarrow a}0$, da 
	$\mu \textgreater 0$ beliebig.\\
	\strut\hfill$\square$\\
	\textbf{5.13. \underline{Illustration der Kettenregelformel:}} $D(g\circ 
	f)(a)=$\setulcolor{green}\ul{$(Dg)(f(a))$}$\cdot$\setulcolor{magenta}\ul{$(Df)(a)$}\\
	$\begin{pmatrix}
		D_1(g\circ f)_1(a)&\cdots&D_n(g\circ f)_1(a)\\
		\vdots& &\vdots\\
		D_1(g\circ f)_k(a)&\cdots&D_n(g\circ f)_k(a)
	\end{pmatrix}=\begin{pmatrix}
	D_1g_1(f(a))&\cdots&D_mg_1(f(a))\\
	\vdots& &\vdots\\
	D_1g_k(f(a))&\cdots&D_mg_k(f(a))\\
	\end{pmatrix}=\begin{pmatrix}
		D_1f_1(a)&\cdots&D_nf_1(a)\\
		\vdots& &\vdots\\
		D_1f_m(a)&\cdots&D_nf_m(a)\\
	\end{pmatrix}.$\\
	\\
	\textbf{5.14. \underline{$\bullet$ Bsp.:}} Sei $a\in U, T_a: 
	\mathbb{R}^n\rightarrow\mathbb{R}^n, x\rightarrowtail x+a,$ ist diff'bar. 
	\setulcolor{red}(\ul{Translation um a})\\
	\begin{align}
		\text{Dann: f in a diff'bar} &\Leftrightarrow f \circ T_{-a} \text{in o 
		diff'bar}\\
		&\Leftrightarrow T_{-f(a)} \circ f \text{in a diff'bar.}
	\end{align}
	$\bullet$\underline{Bsp.:} 
	$\mathbb{R}^3\xrightarrow{f}\mathbb{R}^2\xrightarrow{g}\mathbb{R}, 
	\mathbb{R}^3\xrightarrow{g\circ f}\mathbb{R}, n=3, m=2, g=1, 
	f(x,y,z)=\begin{pmatrix}
		x^{\textcolor{blue}{2}}+y\\
		2y-z
	\end{pmatrix}, g(u,v))uv$\\
Betr. $a=\begin{pmatrix}
	1\\2\\3
\end{pmatrix}\in \mathbb{R}^3.$ Dann:$ D(g\circ 
f)(a)=(Dg)\underbrace{f(a)}_{=\begin{pmatrix}
	3\\1
\end{pmatrix}}\cdot(Df)\underbrace{(a)}_{=\begin{pmatrix}
	\textcolor{red}{1}\\2\\3
\end{pmatrix}}=(1,3)\cdot\begin{pmatrix}
	\textcolor{blue}{2}\cdot\textcolor{red}{1}&1&0\\
	0&2&-1
\end{pmatrix}=(2,7,-3)\in\mathbb{R}^{1 x 3}.$\\
\\
\textbf{5.15. \underline{Bsp.:}} $n=k=1, m=3: f(t)=\begin{pmatrix}
	\varphi(t)\\\uppsi(t)\\\chi(t)
\end{pmatrix}$ mit $\varphi, \uppsi, \chi:U\rightarrow\mathbb{R}, a\in U 
\subseteq\mathbb{R}, \varphi,\uppsi,\chi$ diff'bar in $a\in\mathbb{R}.$\\
sei $b:f(a), g:U_1\rightarrow, U_1\subseteq \mathbb{R}^3,$ b innerer Punkt von 
$U_1, h:=g\circ f.$\\
Somit zeigt die Kettenregel, dass\\
\begin{align}
	h'(a)&=g'(b)\cdot f'(a)=(D-1g(f(a)), D-2g(f(a)),D_3g(f(a)))\cdot 
	\begin{pmatrix}
		\varphi'(a)\\\uppsi'(a)\\\chi'(a)
	\end{pmatrix}\\
	&D_1g(f(a))\varphi'(a)+D_2g(f(a))\uppsi'(a)+D_3 g(f(a))\chi'(a)\in 
	\mathbb{R}
\end{align}\\
In der Literatur wird dafür oft geschrieben: (mit 
$x(t)=\varphi(t),y(t)=\uppsi(t), z(t)=\chi(t)$)\\
$\frac{dh}{dt}=\frac{\delta f}{\delta x}\frac{d x}{d t}+\frac{\delta g}{\delta 
y}\frac{dy}{dt}+\frac{\delta g}{\delta z}\frac{dz}{dt}$ oder auch 
$dh=\frac{\delta g}{\delta x}dx+\frac{\delta g}{\delta y}dy+\frac{\delta 
g}{\delta z}dz.$\\
\\
\textbf{5.16.} Spezialfall k=1 der Kettenregel, Verallgemeinerung von 
\setulcolor{blue} \ul{5.15}:\\
$\frac{\delta g}{\delta t_j}(a)=\sum_{i=1}^{m}\frac{\delta g}{\delta 
x_i}(f_1(a),...,f_m(a))\cdot\frac{\delta f_i}{\delta t_j}(a)$ für alle 
$j\in\{1,...,n\},$ bzw. schreibbar als $D_j(g\circ 
f)(a)=(D_1g(f(a)),...,D_mg(f(a)))\cdot(d_jf_i(a))_{1\leq i\leq m}.$\\
Bildet man rechts das Matrixprodukt, so ist dies $=\sum_{i=1}^{m} 
D_ig(f(a))\cdot D_jf_i(a).$\\
Ist $k=n=1$, folgt $D(g\cdot f)(a)=\textless(grad g)\circ f,f'\textgreater 
(a)$, vgl. \ul{5.15}.\\
\\
\textbf{5.17.} Berechnung von Richtungsableitungen im Fall 
\underline{\underline{m=1:}}\\
\ul{Satz 5.3.} kann mir der Kettenregel bewiesen werden:\\
\textopencorner Haben $D_vf(a)=\lim\limits_{h\rightarrow 
0}\frac{1}{h}(f(a+hv)-f(a))=g'(0)$\\
für die Funktion $g(h):=f(a+hv)=f\circ s(h),$\\
wo $s(h):=a+hv, s:\mathbb{R}\rightarrow\mathbb{R}^n.$\\
Die \ul{Kettenregel} liefert $D_vf(a)=g'(0)=D(f\circ s)(0)=(Df)(s(0))\cdot 
s'(0)=Df(a)\cdot v\textcorner.$\\
\\
\textbf{5.18 \underline{Bem.:}} Die Voraussetzung "f total diff'bar" in 
\ul{5.3} ist notwendig!\\
Betr. z.b. $f:\mathbb{R}^2\rightarrow \mathbb{R}, f(x,y)=\begin{cases}
	\frac{x^2y}{x^4+y^2} &\text{für }(x,y)\neq (0,0),\\
	0&\text{sonst.}
\end{cases}$\\
	Dann ist 
	($D_{\underbrace{u,v}_{\text{Richtungsvektor}}}f)(0,0)=\frac{u^2}{v}$ für 
	$v\neq0$, denn 
	$\frac{1}{t}(f(\underbrace{(0,0)+t\cdot(u,v)}_{=(tu,tv)})-f(0,0))= 
	\frac{u^2v}{t^2u^4+v^2} 
	\xrightarrow{t\rightarrow0}\underbrace{\frac{u^2}{v}}_{\neq 0},$\\
	und $D_1f(0,0)=D_2f(0,0)=0,$ also grad $f(0,0)=\begin{pmatrix}
		0\\0
	\end{pmatrix},$ und die r.f. in \ul{5.3} ist =0.\\
\end{document}