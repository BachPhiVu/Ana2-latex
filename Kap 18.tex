\documentclass[]{scrartcl}
\title{Vorlesung Analysis II}
\usepackage{amsmath,amssymb,amsfonts}
\usepackage{stmaryrd}
\usepackage{mathtools}
\usepackage{latexsym}
\usepackage{graphicx}
\usepackage{tikz}
\usepackage{xcolor}
\usepackage[most]{tcolorbox}
\usepackage{soul}
\usepackage{ upgreek }
\usepackage{hyperref}
\usepackage{tipa}
\usepackage[dvipsnames]{xcolor}
\hypersetup{
	colorlinks=true,
	linkcolor=blue,
	filecolor=magenta,      
	urlcolor=cyan,
	pdftitle={Overleaf Example},
	pdfpagemode=FullScreen,
}
\newcommand{\redcircle}[1]{%
	\tikz[baseline=(char.base)]{
		\node[shape=circle, draw=red, text=red, thick, inner sep=1pt] (char) 
		{\textbf{#1}};
	}%
}
\newcommand{\bluecircle}[1]{%
	\tikz[baseline=(char.base)]{
		\node[shape=circle, draw=blue, text=blue, thick, inner sep=1pt] (char) 
		{\textbf{#1}};
	}%
}
\newcommand{\blackcircle}[1]{%
	\tikz[baseline=(char.base)]{
		\node[shape=circle, draw=black, text=black, thick, inner sep=1pt] 
		(char) 
		{\textbf{#1}};
	}%
}
\newcommand{\orangecircle}[1]{%
	\tikz[baseline=(char.base)]{
		\node[shape=circle, draw=orange, text=orange, thick, inner sep=1pt] 
		(char) 
		{\textbf{#1}};
	}%
}
\newcommand{\redul}[1]{\setulcolor{red}{\ul{#1}}}
\newcommand{\blueul}[1]{\setulcolor{blue}{\ul{#1}}}
\newcommand{\yelul}[1]{\setulcolor{yellow}{\ul{#1}}}
\newcommand{\greenul}[1]{\setulcolor{green}{\ul{#1}}}
\newcommand{\oraul}[1]{\setulcolor{orange}{\ul{#1}}}
\setul{1pt}{3pt} % Linienhöhe und Abstand zum Text (optional anpassbar)

\setlength{\topmargin}{-.5in} \setlength{\textheight}{9.25in}
\setlength{\oddsidemargin}{0in} \setlength{\textwidth}{6.8in}
\setlength{\parindent}{0pt}

\begin{document}
	\maketitle
	\textbf{\underline{Teil 3: Gewöhnliche Differentialgleichungen}}\\
	\\
	\textbf{\underline{an 18: Lineare DGL 1. Ordnung}}\\
	\\
	\textbf{\underline{\underline{Stichworte:} Variation der Konstanten, zugeh. homogene DGL, partikuläre Lsg.}}\\
	\\
	\textbf{\underline{Literatur:}} \blueul{[Hoffmann], kapitel 7.3.}\\
	\\
	\textbf{18.1. \underline{Einleitung:}} Bereits die einfache DGL $y'=\alpha y$ beschreibt exponentielles Verhalten (Wachstum für $\alpha\textgreater0$, zerfall für $\alpha\textless0$), in vielen Anwendungen ein Standardkonzept. Wir behandeln die DGL $y'=f(x)y+g(x)$ als Verallgemeinerung dieser Form.\\
	\\
	\textbf{18.2. \underline{Motivation:}} Die Lineare DGL 1.Ordnung wird untersucht.\\
	\\
	\textbf{18.3. \underline{Vereinbarung:}} Betr. die DGL \fcolorbox{red}{white}{$y'=f(x)y+g(x)$}\blackcircle{*}\\
	wo $f,g:j\rightarrow\mathbb{R}$ stetig, $j\subseteq \mathbb{R}$ ein IV. Die r.s. ist linear in y.\\
	\\
	\textbf{18.4. \underline{Bem.:}} Für $a\in j$ wird durch \greenul{$y_0(x):= \exp(\int_{a}^{x}f(t)dt)$}, $x\in j$, \greenul{eine Lsg. $y_0$} der \redul{zugehörigen homogenen (linearen) DGL} auf j erklärt, die $y_0(x)\neq 0, y_0(a)=1$ erfüllt.\\
	\fcolorbox{red}{white}{$f'=f(x)y$}$\blackcircle{*}_l$\\
	\\
	\textbf{18.5. Satz:} $\bullet$ Für $a \in j$ und $b\in\mathbb{R}$ ist \greenul{die (eindeutig bestimmte) Lsg. y von $\blackcircle{*}$ auf j mit y(a)=b gegeben durch}\\
	\greenul{$y(x)=y_0(x)\cdot(\int_{a}^{x}g(t)y_0(t)^{-1}dt+b)$}. \blackcircle{+}\\
	$\bullet$\greenul{Sämtliche Lösungen von $\blackcircle{*}$} erhält man durch \greenul{Variation von a und b} (d.h. a=a(x), b=b(x)) und \greenul{Einschränkung auf Teilintervalle}.\\
	\underline{Beweis:} $\bullet$ Sei y eine Lsg. von $\blackcircle{*}$ in einem IV $j_0$ mit $a \in j_0 \subseteq j$ und $y(a)b \in \mathbb{R}$. Wir schreiben y in der Form \oraul{$y(x)=c(x)y_0(x)$}, $x\in j_0$, \redul{"Variation der Konstanten"}\\
	mit $c:j_0\rightarrow\mathbb{R}, x$ (stetig)diff'bar (die Glg. kann als Def. für c gelesen werden).\\
	\\
	Nehmen wir diese Form $y=cy_0$ an, dann gilt damit \\
	\underline{$fcy_0$}+g$= fy+g=y'=c'y_0+cy_0=c'y_0+$\underline{$cfy_0$}\\
	$\Rightarrow c'(t)=g(t)y_0(t)^{-1}$,\\
	somit notwendig $y(x)=y_0(x)\cdot(\int_{a}^{x}g(t)y_0(t)^{-1}dt+b)$, d.h. \blackcircle{+}.\\
	$\bullet$ Andererseits wird durch \blackcircle{+} eine Lsg. von \blackcircle{+} mit y(a)=b erklärt.\\
	\strut\hfill$\square$\\
	\textbf{e18.6. \underline{Folgerung:}} (a) Für die zugeh. \greenul{homogene DGL $\blackcircle{*}_h$} sind \greenul{alle Lsg. auf j} gegeben durch \greenul{$y(x)=by_0(x)$}, $x\in j$, $b\in\mathbb{R}$.\\
	(b) Für eine Lsg. y der \greenul{homogenen DGL $\blackcircle{*}_h$} gilt: \greenul{$y\neq0\Rightarrow\forall x \in j : y(x)\neq0$}.\\
	(c) \greenul{Jede bel. Lsg. von $\blackcircle{*}$} auf j entsteht \greenul{aus einer speziellen} ("\redul{partikulären"})\greenul{Lsg. durch Addition eine Lsg. der homogenen DGL $\blackcircle{*}_h$}.\\
	\underline{Bew.:} (a): direkt ablesbar aus \blackcircle{+} mit g(t):=0, $t\in j$.\\
	(b): aus (a), da $y_0\neq 0$ für $x\in j$.\\
	(c): aus der Linearität der Ableitung folgt:\\
	Sind y,z Lsgn. von $\blackcircle{*}$, so gilt \oraul{$(y-z)'=y'-z'=f(y)-f(y)=f(y-z)$}.\\
	Also ist y-z Lsg. von $\blackcircle{*}_h$, und \oraul{y=z+(y-z)} die gewünschte Darstellung.\\
	\strut\hfill$\square$\\
	Die hier enthaltenen Linearitätsüberlegungen sind aus der Linearen Algebra 
	bereits bei der Lösung Linearer Gleichungssysteme bekannt:\\
	\textbf{18.7. \underline{Bem.:}} (a) \greenul{u,v Lsgn. von 
	$\blackcircle{*}_h$} $\Rightarrow$ \greenul{$\alpha u+\beta v$ Lsg. von 
	$\blackcircle{*}_h$} für alle $\alpha, \beta \in \mathbb{R}$\\
	(b) \greenul{u Lsg. von $\blackcircle{*}$} $\wedge$ \greenul{v Lsg. von 
	$\blackcircle{*}_h$}\\
	(c) \greenul{u,v Lsg. von $\blackcircle{*}$} $\Rightarrow$ \greenul{u-v 
	Lsg. 
	von $\blackcircle{*}_h$}\\
	Die Beh. (a) zeigt, dass die \greenul{Menge der Lsgn. der homogenen DGL 
	$\blackcircle{*}_h$} bereits einem \greenul{$\mathbb{R}$-Vektorraum} 
	liefert.\\
	\underline{Bew.:} (c): siehe \blueul{18.6.(c)}.\\
	\underline{(a),(b)}: ebenso aus der \oraul{Linearität der Ableitung}:\\
	\underline{(a)}: $(\alpha u+\beta v)'=\alpha u'+\beta v'=\alpha f(u)+\beta 
	f(v)=f(\alpha u+ \beta v)$,\\
	\underline{(b)}: $(u+v)'=u'+v'=(f(u)+g)+f(v)=f(u+v)+g.$\\
	\strut\hfill$\square$\\
	\textbf{18.8. \underline{Bsp.:}} DGL \fcolorbox{red}{white}{y'=-xy+3x}, 
	\redul{y(0)=5}.\\
	$\bullet$ Zur Lsg. dieser AWA ist in \blueul{Satz 18.5.} zu setzen:\\
	$j:=\mathbb{R}, a:=0, b:=5, f(x):=-x, g(x):=3x.$\\
	Es ergibt sich: 
	\greenul{$y_0(x)$}:=$\exp(\int_{0}^{k}(-t)dt)=$\greenul{$\exp(-\frac{1}{2}x^2)$}\\
	\greenul{y(x)}:=$y_0(x)\cdot (\int_{0}^{x}3t 
	\exp(-\frac{1}{2}t^2)dt+5)=$\greenul{$y_0(x)\cdot(3\exp(\frac{1}{2}x^2)+2)$},\\
	wegen $\int_{0}^{x}3t \exp(\frac{1}{2}t^2)dt=3\exp(\frac{1}{2}t^2) 
	\underset{0}{\overset{x}{\rule{0.4pt}{0.5cm}}}=3(\exp(\frac{1}{2}x^2)-1)$\\
	Daher ist \greenul{$y(x)3+2\exp(-\frac{1}{2}x^2)$} die eindeutig bestimmt 
	\greenul{Lsg. der AWA}.\\
	$\bullet$ Dieselbe AWA \redul{direkt mit "Variation der Konstanten"} gelöst 
	(ohne Formel\blackcircle{+}):\\
	$y_0(x)=\exp(-\frac{1}{2}x^2)$ erfüllt $y_0'=-xy_0, y_0(0)=1.$\\
	Der \yelul{Ansatz} \redul{$y(x)=c(x)y_0(x)$} liefert\\
	\underline{$-x(cy_0)$}+$3x=-xy+3x=y'=c'y_0+cy_0'=c'y_0+$\underline{$c(-xy_0)$}\\
	$\Rightarrow c'y_0=3x \Rightarrow c'(x)=3x\exp(\frac{1}{2}x^2)$.\\
	Daraus folgt $c(x)=3\exp(\frac{1}{2}x^2)+\alpha$. DIe Anfangswertbedingung 
	c(0)=y(0)=5 gibt dann $\alpha$=2, zusammen also wieder die Lsg. 
	$y(x)=3+2\exp(-\frac{1}{2}x^2).$\\
	\\
	$\bullet$ Oft ist es noch einfacher, eine \underline{Lsg. von 
	\blackcircle{*} zu erraten} und dann mit \blueul{Satz 18.6.(c)(und (a))} 
	die allgemeine Lsg. zu notieren:\\
	\\
	Schreibt man die geg. DGL in der Form $y'=x(-y+3)$, so erkennt man leicht 
	die Konstante Fkt. \redul{$y_p(x):=3$} als \redul{partikuläre Lösung}.\\
	\\
	Mit der oben schon besimmten Lsg. $y_0$ der zugeh. homogenen DGL ist die 
	\redul{allgemeine Lösung} (nach \blueul{Satz 18.6.(a) und (c)}) dann\\
	\\
	$y(x)=by_0(x)+y_p(x)=b\exp(-\frac{1}{2}x^2)+3$, mit $b\in\mathbb{R}$ bel.\\
	Die Forderung $y(0)=5$ zeige dann abschließend b=2.
\end{document}