\documentclass[]{scrartcl}
\title{Vorlesung Analysis II}
\usepackage{amsmath,amssymb,amsfonts}
\usepackage{stmaryrd}
\usepackage{mathtools}
\usepackage{latexsym}
\usepackage{graphicx}
\usepackage{tikz}
\usepackage{xcolor}
\usepackage[most]{tcolorbox}
\usepackage{soul}
\usepackage{ upgreek }
\usepackage{hyperref}
\usepackage{tipa}
\usepackage[dvipsnames]{xcolor}
\hypersetup{
	colorlinks=true,
	linkcolor=blue,
	filecolor=magenta,      
	urlcolor=cyan,
	pdftitle={Overleaf Example},
	pdfpagemode=FullScreen,
}
\newcommand{\redcircle}[1]{%
	\tikz[baseline=(char.base)]{
		\node[shape=circle, draw=red, text=red, thick, inner sep=1pt] (char) 
		{\textbf{#1}};
	}%
}
\newcommand{\bluecircle}[1]{%
	\tikz[baseline=(char.base)]{
		\node[shape=circle, draw=blue, text=blue, thick, inner sep=1pt] (char) 
		{\textbf{#1}};
	}%
}
\newcommand{\blackcircle}[1]{%
	\tikz[baseline=(char.base)]{
		\node[shape=circle, draw=black, text=black, thick, inner sep=1pt] 
		(char) 
		{\textbf{#1}};
	}%
}
\newcommand{\orangecircle}[1]{%
	\tikz[baseline=(char.base)]{
		\node[shape=circle, draw=orange, text=orange, thick, inner sep=1pt] 
		(char) 
		{\textbf{#1}};
	}%
}
\newcommand{\redul}[1]{\setulcolor{red}{\ul{#1}}}
\newcommand{\blueul}[1]{\setulcolor{blue}{\ul{#1}}}
\newcommand{\yelul}[1]{\setulcolor{yellow}{\ul{#1}}}
\newcommand{\greenul}[1]{\setulcolor{green}{\ul{#1}}}
\newcommand{\oraul}[1]{\setulcolor{orange}{\ul{#1}}}
\setul{1pt}{3pt} % Linienhöhe und Abstand zum Text (optional anpassbar)

\setlength{\topmargin}{-.5in} \setlength{\textheight}{9.25in}
\setlength{\oddsidemargin}{0in} \setlength{\textwidth}{6.8in}
\setlength{\parindent}{0pt}

\begin{document}
	\maketitle
	\textbf{\underline{Teil 2: Topologische Grundbegriffe in metrischen 
	Räumen}}\\
	\\
	\textbf{\underline{an14: Weierstraßscher Approximationssatz}} (WAS)\\
	\\
	\textbf{\underline{\underline{Stichworte:} Supremumsnorm auf Kompakter 
	Menge, WAS mit Polynome}}\\
	\\
	\textbf{\underline{Literatur:}} \blueul{[Königsberger, Analysis 1], Kapitel 
	15/16}.\\
	\\
	\textbf{14.1. \underline{Einleitung:}} Die Bedeutung der Kompaktheit als 
	topologische Eigenschaft ist für die Mathematik nicht zu unterschätzen. In 
	diesem Kapitel zeigen wir als Anwendung davon den Weierstraßschen 
	Approximationssatz (für Polynome und trigonomische Polynome), und insb., 
	dass $\sum_{n=1}^{\infty}\frac{1}{n^2}=\frac{\pi^2}{6}$\\
	\\
	\textbf{14.2. \underline{Motivation:}} Stetige Funktion lassen sich auf 
	Kompakten Mengen beliebig genau durch Polynome approximieren. Dies besagt 
	der WAS.\\
	\\
	\textbf{14.3. \underline{Vereinbarung:}} Sei $K\subseteq \mathbb{R}$ 
	Kompakt, $f \in \phi(k,\mathbb{C}), \epsilon \textgreater0.$\\
	Wie betrachten die gleichmäßige Konvergenz von f durch Funktionenfolgen auf 
	K, zunächst mit Polynomen über $\mathbb{C}$, nähmlich in der Supremumsnorm 
	auf K.\\
	\\
	\textbf{14.4. \underline{Def.:}} Für $f\in\phi (K,\mathbb{C})$ sei 
	$||\cdot||_K:\phi(K,\mathbb{C})\rightarrow\mathbb{R}_{\supseteq0},$ 
	\yelul{$||f||_K$}:=$\sup\{|f(x)|;x \in K\}$\\
	die \redul{Supremumsnorm} auf K.\\
	\\
	\textbf{14.5. \underline{Bem.:}} Weil K Kompakt, ex. $||f||_K$ immer und 
	\greenul{ist stets beschränkt} (d.h. $\exists \mathbb{R}_{\supseteq0}$), 
	und weiter \greenul{ist das Supremum ein Maximum}, vgl. \blueul{an 
	13.13/14}. Das \greenul{Maximum} von $\{|f(x)|; x\in K\}$ ist 
	\greenul{eindeutig bestimmt}, kann aber in verschiedenen Stellen $x\in K$ 
	angenommen werden. Weiter ist \greenul{$||\cdot||_K$ eine Norm}.\\
	\\
	\textbf{14.6. \redul{Weierstraßscher Approximationssatz (WAS):}}\\
	\underline{Vor.:} $K\subseteq \mathbb{R}$ \greenul{Kompakt, 
	$f\in\phi(K,\mathbb{C})$}, $\epsilon\textgreater 0.$\\
	\underline{Beh.:} \greenul{$\exists p \in \mathbb{C}[X]: ||f-p||_K 
	\textless \epsilon$}.\\
	\\
	Einige Vorbereitungen zum Beweis, der in \blueul{14.11.} geführt wird.\\
	\textbf{14.7. \underline{Def.:}} Sei \yelul{$\overline{\mathcal{O}:= 
	\overline{\mathcal{P}(K)}}$}:=$\{f:K\rightarrow \mathbb{R}; f 
	\text{stetig}, \forall \epsilon \textgreater0 \exists 
	p\in\mathbb{R}[X]:||f-p||_K\textless \epsilon\}$\\
	die \redul{Menge aller (reellen) stetigen Funktionen auf K}, die sich 
	\redul{auf K beliebig genau durch (reelle) Polynome approximieren lassen}.\\
	\\
	Zunächst analysieren wir $\overline{\mathcal{P}}$:\\
	\textbf{14.8. \underline{Hilfssatz:}} $\overline{\mathcal{P}}$ ist eine 
	\greenul{Unteralgebra von $\phi(K,\mathbb{R})$},\\
	d.h. $\overline{\mathcal{P}}$ ist \greenul{UVR} und \greenul{$f,g \in 
	\overline{\mathcal{P}} \Rightarrow f\cdot g \in \overline{\mathcal{P}}$}.\\
	\underline{Bew.:} $f,g\in \overline{\mathcal{P}}, \epsilon 
	\textgreater0\Rightarrow \exists p,q \in \mathbb{R} [X]: ||f-p||_K\textless 
	\epsilon$.\\
	Dann:\\ 
	\begin{align}
		||(f+g)-(p+q)||_K &\leq ||f-p||_K +||g-q||_K\leq2\epsilon\\
		\text{und} ||f\cdot g-p\cdot q||_K&\leq 
		||f-p||_K\cdot||q||_K+||g-q||_K\cdot||f||_K\\
		&\leq \epsilon (||q||_K +||f||_K) \leq \epsilon(\epsilon + 
		\underbrace{||g||_K+||f||_K}_{\text{beschränkt, s. 
		\blueul{14.5.}}})\xrightarrow{\epsilon \rightarrow0}0.
	\end{align}\\
	\strut\hfill$\square$\\
	\textbf{14.9. \underline{Hilfssatz:}} $f,g \in 
	\overline{\mathcal{P}}\Rightarrow$ \greenul{$|f|, \max(f,g), \min(f,g)\in 
	\overline{\mathcal{P}}$}.\\
	\underline{Bew.:} Es ist $\max(f,g)=\frac{1}{2}(f+g+|f-g|),\\
	\min(f,g)=\frac{1}{2}(f+g-|f-g|),$\\
	also \oraul{gen. z.z.: $\overline{\mathcal{P}}\ni|f|,$ falls $f\in 
	\overline{\mathcal{P}}$}.\\
	Dazu sei \OE $f\neq 0 \Rightarrow ||f||_K \neq 0,$ betr. also 
	\oraul{$\phi:=$}$\frac{f}{||f||_K}$, 
	\oraul{z.z.}:$|\phi|\in\overline{\mathcal{P}}.$\\
	\textopencorner Bem.: $|x|\leq 1 \Rightarrow 
	\sqrt{x^2}=|x|=\sum_{n=0}^{\infty}\begin{pmatrix}
		1/2\\n
	\end{pmatrix}(x^2-1)^n$, dies ist klar für $|x|\textless 1$, vgl. 
	\blueul{An 19.20}.\\
	Weiter Konvergiert die \oraul{binomische Reihe gleichmäßig} für $|x|\leq1$, 
	weil $|\begin{pmatrix}
		1/2\\n
	\end{pmatrix}| \leq C\cdot \frac{1}{n\sqrt{n}}.$\textcorner\\
	Hier: $|\phi|\leq1\Rightarrow|\phi|=\sum_{n=0}^{\infty}\begin{pmatrix}
		1/2\\n
	\end{pmatrix}(\phi^2-1)^n$.\\
	Zu $\epsilon \textgreater0$ ex. \oraul{Partialsumme $P_N$} davon mit 
	\oraul{$|| |\phi|-P_N||_K\textless \epsilon$}\\
	$\Rightarrow P_N \in \overline{\mathcal{P}},$ d.h. $\exists p \in 
	\mathbb{R}[X]:||P_N-p||_K\textless\epsilon$\\
	$\Rightarrow$ \oraul{$|||\phi|-p||_K \textless 2\epsilon,$ d.h. $|\phi|\in 
	\overline{\mathcal{P}}$}.\\
	\strut\hfill$\square$\\
	\\
	\textbf{14.10. \underline{Hilfssatz:}} $f \in \phi(K, \mathbb{R}),$ 
	gegebenes $x\in K, \epsilon\textgreater0.$\\
	\underline{Beh.:} \greenul{$\exists q \in \overline{\mathcal{P}}:$} (a) 
	\greenul{$q(x)=f(x)$}\\
	 (b) \greenul{$q\leq f+\epsilon$ auf K}.\\
	 \underline{Bew.:} $\forall z \in K$ wähle \oraul{affinlineares 
	 $l_z(y)=ay+b$}, also \oraul{$l_z \in \overline{\mathcal{P}}$},\\
	mit \oraul{$l_z(x)=f(x), l_z(z)=f(z)$}.\\
	$l_z$ stetig $\Rightarrow \in$ offenes $IV$ \oraul{$I_Z\ni Z$} mit 
	\oraul{$l_Z \leq f + \epsilon$ auf $I_Z$}.\\
	Ferner: $K\subseteq \bigcup_{z \in K}I_Z$ ist \oraul{überdeckung} 
	$\Rightarrow \in z_1,...,z_n: K \subseteq \bigcup_{j=1}^{n}I_{z_j}$.\\
	Sei \oraul{q:=min$\{l_{z_i}$}$;j\in \{1,...,n\}\}$. Dann gilt:\\
	q erfüllt \blueul{(a) und (b)}, denn jedes $z\in K$ \oraul{liegt in einem 
	$I_{z_j}$}. Ferner \oraul{$q\in\overline{\mathcal{P}}$} nach 
	\blueul{Hilfssatz 14.9.}\\
	\\
	\textbf{14.11. \underline{Beweis des Weierstraßschen 
	Approximatinssatzes:}}\\
	\oraul{1. Fall $\mathbb{K}=\mathbb{R}$}, d.h. 
	$\mathbb{K}\subseteq\mathbb{R}$ Kompakt.\\
	$\forall x\in K$ wähle $q_x \in \overline{\mathcal{P}}$ nach 
	\blueul{Hilfssatz 14.10.}\\
	$q_x$ stetig $\Rightarrow \exists$ offenes IV $I_x \ni x, f-\epsilon \leq 
	q_x$ auf $I_x$ \textopencorner wg. \blueul{14.10a}\textcorner\\
	K Kompakt $\Rightarrow \exists x_1,...,x_n$ mit $K\subseteq \bigcup_{j=1}^{n} I_{x_j}.$\\
	Sei $g:=\max_{j\in\{1,...,n\}} q_{x_j}\Rightarrow g \geq f-\epsilon$ auf ganz K.\\
	$\xRightarrow{14.10 (b)} g \leq f+\epsilon,$ also $||f-g||_K\leq \epsilon,$ d.h. $g\in\overline{\mathcal{P}}$ nach \blueul{Hilfssatz 14.9.}\\
	$\Rightarrow\exists p \in \mathcal{P}:||g-p||_k\textless \epsilon$\\
	$\Rightarrow ||f-p||_K\leq ||f-g||_K+||g-p||_k\textless2\epsilon$.\\
	\oraul{2. Fall $\mathbb{K}=\mathbb{C}$}, d.h. $K\subseteq\mathbb{C}$ Kompakt.\\
	Wähle $u,v\in \mathcal{P}(\mathbb{R})$ mit \oraul{$||\mathcal{P}e f-u||_k+\textless \epsilon$} und \oraul{$||\mathcal{Y}mf-v||_K ßtextless \epsilon$}.\\
	Sei \oraul{p:=u+iv} $\in \mathcal{P}(\mathbb{C})$, wo $i=\sqrt{-1}.$\\
	$\Rightarrow ||f-p||_K \leq ||\mathcal{P}e f-u||_K+||\mathcal{Y}mf-v||_K\textless 2\epsilon$.\\
	\strut\hfill$\square$\\
	\textbf{14.12. \underline{Bem.:}} Der \blueul{WAS 14.6.} zeigt, dass Potenzreihen gleichmäßig Konvergieren auf Kompakten Mengen innerhalb des Konvergenzintervalls (Fall $\mathbb{R}$) bzw. Konvergenzkreises (Fall $\mathbb{C}$). Auf dem offenen Kgz. IV/Kreis liegt i.a. Keine Kgz. vor.\\
	\\
	\redul{Der WAS für 2$\pi$-periodische Funktionen.}\\
	\textbf{14.12. \underline{Def.:}} $\bullet$ f heißt \redul{2$\pi$-periodisch}:$\Leftrightarrow \forall x: f(x+2\pi)=f(x).$\\
	$\bullet$ Ein \redul{trigonomisches Polynom} vom $Grad \leq n$ ist ein Term vor der Form \yelul{T(x):}=$\sum_{k=-n}^{n}c_ke^{ikx},$ die $c_k\in\mathbb{C}.$\\
	\\
	\textbf{14.13.} Diese \greenul{bilden einen $\mathbb{C}-VR$} und haben eine \redul{cos-sin-Darstellung:}\\
	Löse Gleichungssystem für $a_k, b_k$ mit $k\geq 1$:\\
	$c_0=\frac{a_0}{2}, c_k=\frac{1}{2}(a_k-ib_k), c_{-k}=\frac{1}{2}(a_k+ib_k),$\\
	also $a_0=2c_0, a_k=c_k+c_{-k},b_k=i(c_k-c_{-k})$, es folgt:\\
	\greenul{T(x)}= $\frac{a_0}{2}+\sum_{k=1}^{n}(a_k \cos kx+b_k \sin kx).$\\
	\\
	\textbf{14.14. \underline{Beh.:}}$\bullet$ Die \greenul{Koeffizienten $c_k$} in T (und damit $a_n und b_k$) sind \greenul{eindeutig}.\\
	$\bullet$ Konsequenz: \greenul{T reellwertig $\Leftrightarrow\forall k \overline{c_k}=-c_k\Leftrightarrow \forall k: a_k,b_k \in \mathbb{R}$}.\\
	\underline{Bew.:} $\bullet$ $c_k = \frac{1}{2\pi}\int_{0}^{2\pi}T(x)e^{-ikx}dx$, da: $\frac{1}{2\pi}\int_{0}^{2\pi}e^{ilx}dx=\begin{cases}
		1, l=k\\
		0, l\neq k
	\end{cases}$.\\
	$\bullet$ T reellwertig $\Rightarrow$ $\overline{c_k}=\frac{1}{2\pi}\int_{0}^{2\pi}T(x)e^{ikx}dx=c_{-k}\Rightarrow a_k=\mathcal{P}e c_k,b_k=\mathcal{Y}m c_k$.\\
	\strut\hfill$\square$\\
	\textbf{14.15. \redul{WAS für 2$\pi$-periodische Funktionen:}}\\
	\underline{Vor.:} $f\in \phi(\mathbb{R},\mathbb{C}), f$\greenul{2$\pi$-periodisch}.\\
	\underline{Beh.:} $\forall \epsilon0$ \greenul{$\exists$ trigon. Pol. T: $||f-T||_{\sup} \textless \epsilon$}.\\
	\underline{Bew.:} Sei \yelul{$\overline{\mathcal{T}}$}:= $\{f:\mathbb{R}\rightarrow\mathbb{R} \text{stetig, }2\pi-\text{periodisch, appr'bar durch trigon. Pol.}\}.$\\
	(a) \oraul{$\overline{\mathcal{T}}$ ist Unteralgebra con $\phi(\mathbb{R},\mathbb{R})$}\\
	(b) \oraul{$f,g \in \overline{\mathcal{T}} \Rightarrow |f|, \max(f,g), \min(f,g)\in\overline{\mathcal{T}}$}\\
	(c) \oraul{f wie in WAS}, $x\in\mathbb{R} \Rightarrow$ \oraul{$\exists T \in \overline{\mathcal{T}}$}: $\alpha$\oraul{$T(x)=f(x)$}, $\beta$ \oraul{T$\subseteq f+\epsilon$ auf $\mathbb{R}$}\\
	Hier: ersetzen $l_z$ durch \oraul{trgon. Pol. $L_z$} mit \oraul{$L_z(x)=f(x), L_z(z)=f(z)$}.\\
	Beweis dann analog wie in \blueul{14.11.}\\
	\\
	\textbf{14.15. \underline{Bem.:}} Dieser Satz ist der Ausgangspunkt der Theorie der \blueul{Fourierreihen}, man kommt so zur \blueul{Fourieranalysis}, ein Teilgebiet der Analysis, wo trigonomische Approximationen untersucht werden. Die Fourieranalysis ist auch in physikalischen Anwendungen von Bedeutung.\\
	\\
	Eine Mathematische Anwendung der Fourierreihentheorie:\\
	\textbf{14.16. \underline{Bsp.:}} Sei $f(x)=|x|$ für $-\pi \leq x\leq \pi$, setzee dies $2\pi$-periodisch fort. Ans \blueul{14.13.} erhalten wir die \redul{Fourierreihen $S_\infty f(x)$}:=$\lim\limits_{n\rightarrow\infty}T_n(x)=\frac{a_0}{2}+\sum_{k=1}^{\infty}(a_k \cos kx+b_k \sin kx)$\\
	Da f gerade, sind alle $b_k=0, a_k=\frac{2}{\pi}\int_{0}^{\pi}x\cos(kx)dx, a_0=\frac{2}{\pi}\cdot\frac{\pi^2}{2}=\pi$ \textopencorner\blueul{14.13/14}\textcorner\\
	und für $k\geq1$ ist \\
	$\alpha_k=\frac{2}{\pi}\int_{0}^{\pi}x\cos(kx)dx=(\frac{x}{k}\sin(kx)\underset{0}{\overset{\pi}{\rule{0.4pt}{0.4cm}}}-\frac{1}{k}\int_{0}^{\pi}\sin(kx)dx)$\\
	$=\frac{2}{\pi}\cdot\frac{1}{k^2}\cos(kx)\underset{0}{\overset{\pi}{\rule{0.4pt}{0.4cm}}}=\frac{2}{\pi k^2}(-1+(-1)^k)=\begin{cases}
		\frac{-4}{\pi k^2}, 2tk,\\
		0, 2/k
	\end{cases}.\\
	\Rightarrow$\greenul{$S_\infty f(x)=\frac{\pi}{2}-\frac{4}{\pi}\sum_{n=0}^{\infty}\frac{cos((2n+1)x)}{(2n+1)^2}$}, ist \greenul{glm. Kgt.} auf $\mathbb{R}$ und somit \greenul{stetig}. \textopencorner nach \blueul{An18.9.(b)}\textcorner\\
	Nach dem \blueul{Darstellungssatz 14.18} Kgt. $S_\infty f$ glm. gegen f.\\
	für x=0 erhält man eine Gleichung für $\pi$,\\
	nähmlich $\frac{pi^2}{8}=\sum_{n=0}^{\infty}\frac{1}{(2n+1)^2}$.\\
	Daraus folgt \fcolorbox{green}{white}{$\sum_{n=1}^{\infty}\frac{1}{n^2}=\frac{\pi^2}{6}$}\\, denn 
	$\sum_{n=1}^{\infty}\frac{1}{n^2}=\sum_{n=1}^{\infty}\frac{1}{(2n+1)^2}=\frac{1}{4}\sum_{n=1}^{\infty}\frac{1}{n^2}+\sum_{n=0}^{\infty}\frac{1}{(2n+1)^2}$\\
	$\Rightarrow \frac{3}{4}\sum_{n=1}^{\infty}\frac{1}{n^2}=\frac{\pi^2}{8}\Rightarrow\sum_{n=1}^{\infty}\frac{1}{n^2}=\frac{4}{3}\cdot\frac{\pi^2}{8}=\frac{\pi^2}{6}.$\\
	\\
	\textbf{14.17. \redul{Identitätssatz für Fourierreihen:}}\\
	\underline{Vor.:} $f,g:\mathbb{R}\rightarrow\mathbb{C}$, $f,g 2\pi$-periodisch und integrierbar auf $[0,2\pi]$ stetig und \greenul{mit gleichen Fourierkoeffizienten}.\\
	\underline{Beh.:} \greenul{f=g}.\\
	\underline{Bew.:} $h:=f-g$ hat laut Vor. die Fourierkoeff. 0, dann folgt $h\overset{!}{=}0$,\\
	denn: $T_n(x)=\sum_{k=-n}^{n}c_ke^{ikx}\Rightarrow \int_{0}^{2\pi}h T_n(x)dx=\sum_{k=-n}^{n}c_k \underbrace{\int_{0}^{2\pi}b e^{-ikx}dx}_{=0}=0.$\\
	Laut \blueul{WAS 14.15.} folgt $\exists (T_n)\Rightarrow h$.\\
	$\Rightarrow \int_{0}^{2\pi}|h|^2dx = \lim\limits_{n\rightarrow\infty}\int_{0}^{2\pi}hT_ndx=0$. Da h stetig, ist $\overline{h}$ und $|h|^2$ stetig.\\
	Wäre $h\neq0$, folgte $\int_{0}^{2\pi}|h|^2dx\textgreater0 \lightning.$\\
	\strut\hfill$\square$\\
	\textbf{14.18. \redul{Darstellungssatz:}} \underline{Vor.:} f wie in \blueul{14.17.}, \greenul{$S_\infty$ f kgt. glm.} \underline{Beh.:}  \greenul{$S_\infty f\Rightarrow f$}.\\
	\underline{Bew.:} Sei \oraul{g(x)}:= $\lim\limits_{n\rightarrow\infty}\sum_{k=-n}^{n}$\oraul{$c_k l^{ikx}=s_\infty f(x)$}. Nach \blueul{An18.8.(b)} ist g stetig.\\
	Ferner ist der \oraul{l-te Fourierkoeff. von g} gleich $\frac{1}{2\pi}\lim\limits_{n\rightarrow\infty}\sum_{k=-n}^{n}c_k\underbrace{\int_{0}^{2\pi}e^{ikx}e^{-ilx}dx}_7=2\pi\delta_{k,l}$\oraul{$=c_k$},\\
	dem k-ten Fourierkoeff. von f. mit \blueul{14.17} folgt f=g.\\
	\strut\hfill$\square$	
\end{document}