\documentclass[]{scrartcl}
\title{Vorlesung Analysis II}
\usepackage{amsmath,amssymb,amsfonts}
\usepackage{stmaryrd}
\usepackage{mathtools}
\usepackage{latexsym}
\usepackage{graphicx}
\usepackage{tikz}
\usepackage{xcolor}
\usepackage{soul}
\usepackage{hyperref}
\usepackage{tipa}
\usepackage[dvipsnames]{xcolor}
\hypersetup{
	colorlinks=true,
	linkcolor=blue,
	filecolor=magenta,      
	urlcolor=cyan,
	pdftitle={Overleaf Example},
	pdfpagemode=FullScreen,
}
\newcommand{\redcircle}[1]{%
	\tikz[baseline=(char.base)]{
		\node[shape=circle, draw=red, text=red, thick, inner sep=1pt] (char) 
		{\textbf{#1}};
	}%
}
\setul{1pt}{3pt} % Linienhöhe und Abstand zum Text (optional anpassbar)

\setlength{\topmargin}{-.5in} \setlength{\textheight}{9.25in}
\setlength{\oddsidemargin}{0in} \setlength{\textwidth}{6.8in}
\setlength{\parindent}{0pt}

\begin{document}
	\textbf{\underline{Teil 1: Differentialrechnung im $\mathbb{R}^n$}}\\
	\\
	\textbf{\underline{an5: Partielle und totale Ableitungen}}\\
	\\
	\textbf{\underline{\underline{Stichworte:} Funktionalmatrix, Gradient, 
	Kettenregel, Richtungsableitungen}}\\
	\\
	\textbf{\underline{Literatur:}}\setulcolor{blue}\ul{[Hoff], Kapitel 9.4}\\
	\\
	\textbf{5.1. \underline{Einleitung:}} Die totale Ableitung liefert einen 
	einfachen Weg, Richtungsableitungen zu berechnen. Wirdefinieren für m=1 den 
	Gradienten und beweisen die allgemeine Kettenregel.\\
	\\
	\textbf{5.2. \underline{Konvention:}} Betrachten Funktionen 
	$f:U\rightarrow\mathbb{R}^m$ mit $U\subseteq \mathbb{R}^n$.\\
	Sei $a\in U$ ein innerer Punkt von U, d.h. $\exists s\textgreater0: 
	U_a^s\subseteq U.$\\
	Wir bezeichnen die Menge aller Richtungsvektoren $v\in \mathbb{R}^n$ mit 
	\setulcolor{yellow}\ul{$S^{n-1}$}:=$\{v\in \mathbb{R}^n; ||v||_2=1\}$, die 
	(n-1)-dimensionale \setulcolor{red}\ul{Sphäre im $\mathbb{R}^n$}.\\
	$\bullet$ Für einen inneren Punkt $a\in U$ und ein $v\in S^{n-1}$ haben wir 
	in \setulcolor{blue}\ul{an4.8}\\
	$D_vf(a):=\lim\limits_{0\neq t\rightarrow0}\frac{1}{t}(f(a+tv)-t(a))$\\
	als Richtungsableitung von f in a in Richtung v definiert.\\
	(Diese Def. benutzt, dass $a+t v\in U$ ist aööe $t\in\mathbb{R}$ mit 
	hinreichend kleinen $|t|$)\\
	$\bullet$\underline{Speziell:} Ist $v=e_i$ der i-te Einheitsvektor, so ist\\
	$D_if(a)=\frac{\delta f}{\delta \xi_i}(a)=f_{\xi_i}(a):=(D_{e_i}f)(a)$ die 
	i-te partielle Ableitung.\\
	\\
	\textbf{5.3. \underline{Satz:}} \underline{Vor.:} \setulcolor{green}\ul{f 
	in a diff'bar} (d.h. total diff'bar), \ul{$v\in S^{n-1}$}.\\
	\underline{Beh.:} f ist \ul{in Richtung v in a diff'bar} und 
	\ul{($D_vf$)(a)}=\ul{$\underbrace{(Df)(a)}_{\in\mathbb{R}^{m x n}}\cdot 
	\underbrace{v}_{\in\mathbb{R}^n}$}=$\underbrace{f'(a)(v)}_{\text{mit} 
	f'(a)\in Hom(\mathbb{R}^n,\mathbb{R}^m)\text{ausgedrückt, dh als lineare 
	Abb.}}$.\\
	\underline{Bew.:}$\frac{1}{t}(f(a+tv)-f(a))=\frac{1}{t}(f'(a)(tv)+o(||tv||_2)
	=\underline{Bew.:}\frac{1}{t}(f(a+tv)-f(a))=\frac{1}{t}(f'(a)(tv)+o(||tv||_2))=
	 f'(a)(v)+\underbrace{\underbrace{\frac{|t|\cdot||v||_2}{t}}_{\text{beschränkt}}
	  \cdot 
	 \underbrace{\frac{o(||tv||_2)}{||tv||_2}}_{\xrightarrow{t\rightarrow0}0}}_{\xrightarrow{t\rightarrow0}0}
	$.\\
	\strut\hfill$\square$\\
	
	
	
	
	
	
	
	
\end{document}