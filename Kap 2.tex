\documentclass[]{scrartcl}
\title{Vorlesung Analysis II}
\usepackage{amsmath,amssymb,amsfonts}
\usepackage{mathtools}
\usepackage{graphicx}
\usepackage{tikz}
\usepackage{xcolor}
\usepackage{soul}
\usepackage{hyperref}
\hypersetup{
	colorlinks=true,
	linkcolor=blue,
	filecolor=magenta,      
	urlcolor=cyan,
	pdftitle={Overleaf Example},
	pdfpagemode=FullScreen,
}
\newcommand{\redcircle}[1]{%
	\tikz[baseline=(char.base)]{
		\node[shape=circle, draw=red, text=red, thick, inner sep=1pt] (char) 
		{\textbf{#1}};
	}%
}
\setul{1pt}{3pt} % Linienhöhe und Abstand zum Text (optional anpassbar)

\setlength{\topmargin}{-.5in} \setlength{\textheight}{9.25in}
\setlength{\oddsidemargin}{0in} \setlength{\textwidth}{6.8in}
\setlength{\parindent}{0pt}

\begin{document}
\maketitle
\underline{Teil 1: Differnetialrechnung im $\mathbb{R}^n$}\\
\\
\underline{an2: Geometrie von Funktionen $\mathbb{R}^n\rightarrow\mathbb{R}^m$ 
mit m=1 und n=1}\\
\underline{Stichworte:}Affine Räume, Parameter- und Normdarstellung, Funktionen 
$\mathbb{R}^n\rightarrow\mathbb{R}^m$\\
\underline{Literatur:}\setulcolor{blue} \ul{[Hoff], Kapitel 9.2}\\
\\
2.1\underline{Einleitung}: Nach Kurzer Überlegung zur Darstellung 
affin-Linearer Objekte im $\mathbb{R}^n$, also Geraden, Ebenen, Hyperebenen,... 
arbeiten wir an der geometrischen Anschauung von Funktionen 
$f:\mathbb{R}^n\rightarrow\mathbb{R}^m$, die affinlinear oder nicht affinlinear 
sind. Wir betrachten insbesondere $\mathbb{R}$-wertiger (auch: reellwertiger)\\
Funktionen, d.h. solche Funktionen $f:\mathbb{R}^n\rightarrow\mathbb{R}$ mit n 
= = 1, sowie auch "Kurvenartige Funktionen $f: 
\mathbb{R}\rightarrow\mathbb{R}^m$ mit m = 1.\\\\
2.2\underline{Affine Räume} im $\mathbb{R}^n$: Ist $U\subseteq\mathbb{R}^n$ ein 
Untervektorraum des $\mathbb{R}^n$, so heißt a+U für ein $a\in \mathbb{R}^n$ 
ein (d-dimensionaler) \setulcolor{red} \ul{affiner Raum}, wenn dim U=d ist. 
(Man kann a einen \ul{Aufpunkt} von a+U nennen.)\\
Es gibt folgende Atren zur Beschreibung der El. von a+U:\\
2.3 $\bullet$\underline{Parameterfarstellung:} \ul{Ist u die} Lineare Hülle von 
Vektoren $v_1,...,v_r$, d.h. U= L$(v_1,...,v_r) : = {\alpha_1 v_1+...+\alpha_r 
v_r; \alpha_1,...,\alpha_r\in\mathbb{R}} = \mathbb{R}v_1+...+\mathbb{R}v_r$, 
d.h. die Menge aller Linearkombinationen $\sum_{i=1}^{r}\alpha_i v_i$ der 
$v_1,...,v_r$, auch: der \ul{Span} der $v_1,...,v_r$ geschrieben 
\setulcolor{yellow} \ul{$span(v_1,...,v_r)$},\\
bzw. auch: das \setulcolor{red} \ul{Lineare Erzeugnis} der $v_1,...,v_r$ 
geschrieben$\textless v_1,...,v_r\textgreater$($\leftarrow$ keine 
skalarproduktklammern, sondern "Erzeugnissklammern"!)\\
Dann ist a+U = a+L($v_1,...,v_r) = {a+\alpha_1 v_1 +...+ \alpha_r v_r; 
\alpha_1,...,\alpha_r\in \mathbb{R}}$\\

Sind $v_1,...,v_r$ Linear unabhängig, gilt dim(a+U)=dim U = r, die 
$v_1,...,v_r$ heißen dann \ul{Richtungsvektoren}.\\
Für r= dim U = 1 ist die eine \ul{Gerade} $a + \mathbb{R}v_1 = {a+tv_1; 
t\in\mathbb{R}^n}$, "in Richtung" $v_1\in \mathbb{R}^n, v_1\neq 0$, und mit 
Aufpunkt $a \in \mathbb{R}^n$. Für r=dim U = 2  ist dies eine \ul{Ebene} $a + 
\mathbb{R}v_1 + \mathbb{R}v_2 = {a+tv_1+sv2; t,s \in 
\mathbb{R}}\subseteq\mathbb{R}^n$ mit zwei (linear unabh.) Richtungsvektoren 
$v_1,v_2\in \mathbb{R}^n$ und Aufpunkt $a\in \mathbb{R}$. Usw.\\
Eine besonders einfache Darstellung ist im Fall dim U = n-1 möglich, den 
zugehörigen affinen Raum nennen wir eine \ul{Hyperebene in $\mathbb{R}^n$}:\\
\\
2.4$\bullet$ \underline{Normalendarstellung}(einer \ul{Hyperebene} im 
$\mathbb{R}^n$):\\
Sei \setulcolor{yellow} $H_{c,a}:={x\in\mathbb{R}^n|\textless x,c\textgreater = 
\alpha}$ für $c \in \mathbb{R}^n, c\neq 0,$ und $\alpha \in \mathbb{R}.$\\
Sei $p\in H_{c,\alpha}$ irgenein Punkt dieser Menge, d.h. es gelte $\textless 
p,c \textgreater = \alpha$.\\
Dann ist $H_{c,\alpha} = p+U$ mit einem Untervektorraum  
\begin{abstract}
\section{}
\end{abstract}

\section{}

\end{document}
