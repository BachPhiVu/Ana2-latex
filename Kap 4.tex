\documentclass[]{scrartcl}
\title{Vorlesung Analysis II}
\usepackage{amsmath,amssymb,amsfonts}
\usepackage{stmaryrd}
\usepackage{mathtools}
\usepackage{latexsym}
\usepackage{graphicx}
\usepackage{tikz}
\usepackage{xcolor}
\usepackage{soul}
\usepackage{hyperref}
\usepackage{tipa}
\usepackage[dvipsnames]{xcolor}
\hypersetup{
	colorlinks=true,
	linkcolor=blue,
	filecolor=magenta,      
	urlcolor=cyan,
	pdftitle={Overleaf Example},
	pdfpagemode=FullScreen,
}
\newcommand{\redcircle}[1]{%
	\tikz[baseline=(char.base)]{
		\node[shape=circle, draw=red, text=red, thick, inner sep=1pt] (char) 
		{\textbf{#1}};
	}%
}
\setul{1pt}{3pt} % Linienhöhe und Abstand zum Text (optional anpassbar)

\setlength{\topmargin}{-.5in} \setlength{\textheight}{9.25in}
\setlength{\oddsidemargin}{0in} \setlength{\textwidth}{6.8in}
\setlength{\parindent}{0pt}

\begin{document}
	\textbf{\underline{Teil 1: Differentialgleichung im $\mathbb{R}^n$}}\\
	\\
	\textbf{\underline{an4: Mehrdimensionales Ableiten}}\\
	\\
	\textbf{\underline{\underline{Stichworte:} Richtungsableitung, partielle Ableitung, totale Ableitung, Klein-o}}\\
	\\
	\textbf{\underline{Literatur:}} \setulcolor{blue}\ul{[Hoff], Kapitel 9.4}\\
	\\
	\textbf{4.1. \underline{Einleitung:}} Wir führen den Differenzierbarkeitsbegriff  für Funktionen $\mathbb{R}^n\rightarrow\mathbb{R}^m$ ein: über Richtungableitungen entlang der Koordinatenachsen gelangen wir zu partiellen Ableitungen. Wir definieren die totale Ableitung und sehen wie mit den partiellen Ableitungen der Komponentenfunktionen berechnet werdeb kann. Die "Linearsierung" von f ergibt also die Matrix $Df(a)\in\mathbb{R}^{m x n}$ so, dass f(x)
	
	
	
	
	
	
\end{document} 