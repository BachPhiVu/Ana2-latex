\documentclass[]{scrartcl}
\title{Vorlesung Analysis II}
\usepackage{amsmath,amssymb,amsfonts}
\usepackage{stmaryrd}
\usepackage{mathtools}
\usepackage{latexsym}
\usepackage{graphicx}
\usepackage{tikz}
\usepackage{xcolor}
\usepackage[most]{tcolorbox}
\usepackage{soul}
\usepackage{ upgreek }
\usepackage{hyperref}
\usepackage{tipa}
\usepackage[dvipsnames]{xcolor}
\hypersetup{
	colorlinks=true,
	linkcolor=blue,
	filecolor=magenta,      
	urlcolor=cyan,
	pdftitle={Overleaf Example},
	pdfpagemode=FullScreen,
}
\newcommand{\redcircle}[1]{%
	\tikz[baseline=(char.base)]{
		\node[shape=circle, draw=red, text=red, thick, inner sep=1pt] (char) 
		{\textbf{#1}};
	}%
}
\newcommand{\bluecircle}[1]{%
	\tikz[baseline=(char.base)]{
		\node[shape=circle, draw=blue, text=blue, thick, inner sep=1pt] (char) 
		{\textbf{#1}};
	}%
}
\newcommand{\blackcircle}[1]{%
	\tikz[baseline=(char.base)]{
		\node[shape=circle, draw=black, text=black, thick, inner sep=1pt] 
		(char) 
		{\textbf{#1}};
	}%
}
\newcommand{\orangecircle}[1]{%
	\tikz[baseline=(char.base)]{
		\node[shape=circle, draw=orange, text=orange, thick, inner sep=1pt] 
		(char) 
		{\textbf{#1}};
	}%
}
\newcommand{\redul}[1]{\setulcolor{red}{\ul{#1}}}
\newcommand{\blueul}[1]{\setulcolor{blue}{\ul{#1}}}
\newcommand{\yelul}[1]{\setulcolor{yellow}{\ul{#1}}}
\newcommand{\greenul}[1]{\setulcolor{green}{\ul{#1}}}
\newcommand{\oraul}[1]{\setulcolor{orange}{\ul{#1}}}
\setul{1pt}{3pt} % Linienhöhe und Abstand zum Text (optional anpassbar)

\setlength{\topmargin}{-.5in} \setlength{\textheight}{9.25in}
\setlength{\oddsidemargin}{0in} \setlength{\textwidth}{6.8in}
\setlength{\parindent}{0pt}

\begin{document}
	\maketitle
	\textbf{\underline{Teil 2: Topollogische Grundbegriffe in metrischen Räumen}}\\
	\\
	\textbf{\underline{an15: Zusammenhang in metrischen Räumen}}\\
	\\
	\textbf{\underline{\underline{Stichworte:} zusammenhängend (zush)$\Leftrightarrow$ wegszush.$\Leftrightarrow$ polygonslzush.}}\\
	\\
	\textbf{\underline{Literatur:}} \blueul{[Königsberger], Kapitel 1.5}\\
	\\
	\textbf{15.1. \underline{Einleitung:}} Der Begriff "zusammenhängend" wird für metrische Räume definiert und mit "wegzusammenhängend" und "polygonalzusammenhängend" identifiziert, was über "Verbindungen" zwischen zwei Punkten erklärt wird.\\
	\\
	\textbf{15.2. \underline{Motivation:}} Es ist zunächst leichter definieren, was "nicht zusammenhängend" ist.\\
	\\
	\textbf{15.3. \underline{Vereinbarung:}} ($R,\delta$) sei metrischer Raum, $M\subseteq R,$ damit ist ($M,\delta_{rM\times M}$) metrischer Raum.\\
	\\
	\textbf{15.4. \underline{Def.:}} R heißt \redul{nicht zusammenhängend} (kurz: \redul{zush.})\\
	:$\Leftrightarrow$ $\exists O_1,O_2 \subset R, O_2\neq \o \neq O_2: R=O_1\dot{\cup}O_2$\\
	R heißt \redul{zush.}:$\Leftrightarrow$ R ? nicht zush.\\
	M heißt \redul{zush.}:$\Leftrightarrow (M,\delta_{rM\times M})$ zush.\\
	\\
	\textbf{15.5. \underline{Satz:}} \underline{Vor.:} $R\xrightarrow{f}S$ \greenul{stetig}, R,S \greenul{metrische Räume, R zush.}\\
	\underline{Beh.:} \greenul{f(R) zush.} "\redul{Bilder zush. Mengen sind zush.}"\\
	\underline{Bew.:} Ann.: $f(R)= S_1\cup S_2$ mit $S_1\cap S_2=\o, S_1,S_2\subset f(R)$,\\
	d.h. $\exists O_1,O_2\subset S$ mit \oraul{$S_1=O_1\cap f(R), S_2=O_2\cap f(R)$}, $O_1\cap O_2=\o$.\\
	Betr. \oraul{$f^{-1}(S_1\cup S_2)$}=$f^{-1}(O_1 \cup O_2)=$\oraul{$f^{-1}(O_1)\cup f^{-1}(O_2)$ offen, =R.}\\
	Da R zush., folgt $f^{-1}(O_1)=\o$ oder $f^{-1}(O_2)=\o$, d.h. \oraul{$S_1=\o vS_2=\o$},\\
	so dass also f(R) zush. ist.\\
	\strut\hfill$\square$\\
	
	
	
	
	
	
	
	
	
	
	
	
	
	
	
	
	
	
	
	
	
\end{document}