\documentclass[]{scrartcl}
\title{Vorlesung Analysis II}
\usepackage{amsmath,amssymb,amsfonts}
\usepackage{stmaryrd}
\usepackage{mathtools}
\usepackage{latexsym}
\usepackage{graphicx}
\usepackage{tikz}
\usepackage{xcolor}
\usepackage[most]{tcolorbox}
\usepackage{soul}
\usepackage{ upgreek }
\usepackage{hyperref}
\usepackage{tipa}
\usepackage[dvipsnames]{xcolor}
\hypersetup{
	colorlinks=true,
	linkcolor=blue,
	filecolor=magenta,      
	urlcolor=cyan,
	pdftitle={Overleaf Example},
	pdfpagemode=FullScreen,
}
\newcommand{\redcircle}[1]{%
	\tikz[baseline=(char.base)]{
		\node[shape=circle, draw=red, text=red, thick, inner sep=1pt] (char) 
		{\textbf{#1}};
	}%
}
\newcommand{\bluecircle}[1]{%
	\tikz[baseline=(char.base)]{
		\node[shape=circle, draw=blue, text=blue, thick, inner sep=1pt] (char) 
		{\textbf{#1}};
	}%
}
\newcommand{\blackcircle}[1]{%
	\tikz[baseline=(char.base)]{
		\node[shape=circle, draw=black, text=black, thick, inner sep=1pt] 
		(char) 
		{\textbf{#1}};
	}%
}
\newcommand{\orangecircle}[1]{%
	\tikz[baseline=(char.base)]{
		\node[shape=circle, draw=orange, text=orange, thick, inner sep=1pt] 
		(char) 
		{\textbf{#1}};
	}%
}
\newcommand{\redul}[1]{\setulcolor{red}{\ul{#1}}}
\newcommand{\blueul}[1]{\setulcolor{blue}{\ul{#1}}}
\newcommand{\yelul}[1]{\setulcolor{yellow}{\ul{#1}}}
\newcommand{\greenul}[1]{\setulcolor{green}{\ul{#1}}}
\newcommand{\oraul}[1]{\setulcolor{orange}{\ul{#1}}}
\setul{1pt}{3pt} % Linienhöhe und Abstand zum Text (optional anpassbar)

\setlength{\topmargin}{-.5in} \setlength{\textheight}{9.25in}
\setlength{\oddsidemargin}{0in} \setlength{\textwidth}{6.8in}
\setlength{\parindent}{0pt}

\begin{document}
	\textbf{\underline{Teil 1: Differentialrechung im $\mathbb{R}^n$}}\\
	\\
	\textbf{\underline{an8: lokale Umkehrbarkeit}}\\
	\\
	\textbf{\underline{\underline{Stichworte:} Kontraktion, Banachscher 
	Fixpunktsatz, Lokale Umkehrbarkeit}}\\
	\\
	\textbf{\underline{Literatur:} \blueul{[Forster, Ende von Kap. 8]}}\\
	\\
	\textbf{8.1. \underline{Einleitung:}} Mit dem Banachschen Fixpunktsatz 
	zeigen wir den Satz über die lokale Umkehrbarkeit als Verallgemeinerung des 
	Satzes von der Ableitung von Umkehrfkt.\\
	\\
	\textbf{8.2. \underline{Motivation:}} Sei $a\in U\subset \mathbb{R}^n, f\in 
	l^1(U,\mathbb{R}^n)$,\\
	d.h. $f_j\in l^1(U,\mathbb{R})$ für alle $j\in \{1,...,n\}$.\\
	Haben: $f':U\rightarrow Hom(\mathbb{R}^n,\mathbb{R}^n)=\mathbb{R}^{nxn}$ 
	ist stetig. (d.h. jedes $f_i'$ bzgl. Norm $||\cdot||_\infty$ und dazu 
	äquivalente Normen).\\
	Als \underline{Kriterium für Invertierbarkeit} ist bekannt.\\
	$A\in Hom(\mathbb{R}^n,\mathbb{R}^n)$ invertierbar $\Leftrightarrow$ 
	\greenul{det $A\neq 0$}.\\
	Jetzt: \greenul{$A=f'(a)$ invertierbar}$\Rightarrow$ f \greenul{nahe a 
	invertierbar}, d.h. $\exists U \in \mathcal{U}_a: f_{rU}$ invertierbar.\\
	\\
	\textbf{8.3. \underline{Def.:}} Sei $f:\mathbb{R}^n\rightarrow\mathbb{R}^n, 
	a\in \mathbb{R}^n$.\\
	Dann: a heißt \redul{Fixpunkt von f}, falls f(a)=a ist.\\
	\\
	\textbf{8.4. \underline{Bsp.:}} $\bullet$ Sei 
	$f:\mathbb{R}^n\rightarrow\mathbb{R}^n, x\rightarrowtail x+c
$ für $c\in \mathbb{R}^n$ fest.\\
Für c=o sind alle x fix, für c$\neq$o ist kein x fix.\\
	$\bullet$ Ist f eine Drehung um o las Drehzentrum, so ist o Fixpunkt und 
	alle x fix, wenn der Drehwinkel Vielfaches von 2$\pi$ ist.\\
	\\
	\textbf{8.5. \underline{Def.:}} Sei 
	$f:\mathbb{R}^n\rightarrow\mathbb{R}^n$. Dannn heißt f 
	\redul{Kontrahierend} (oder \redul{Kontraktion}) \redul{mit 
	Kontraktionsfaktor p}$\in [0,1[$, falls $\forall a,b \in \mathbb{R}^n: 
	||f(a)-f(b)||\leq p||a-b||.$\\
	(Wobei $||\cdot||=||\cdot||_infty$ sei) Beachte p\textless 1!\\
	\underline{Bem.:}\greenul{Jede Kontraktion ist stetig} (Klar per Def.).\\
	\\
	\textbf{8.6. \redul{Banachscher Fixpunktsatz:}} Vor.: $f: U\rightarrow U$ 
	\greenul{Kontrahierend mit Kontraktionsfaktor p}, wo $U\subseteq 
	\mathbb{R}^n$ \greenul{abgeschlossen und beschränkt sei.} (Vgl. später Teil 
	2 dieser Vorlesung.)\\
	Beh.: (a)\greenul{$\exists!$ Fixpunkt a von f}.\\
	(b) Setze $x_0\in \mathbb{R}^n$ beliebig, und \greenul{$x_k+1:=f(x_k)$} für 
	alle $k\geq0.$\\
	Dann: \greenul{$||x_k-a||$}$\leq$\greenul{$\frac{p^k}{1-p}||x_1-x_0||$}, 
	d.h. \greenul{$\lim\limits_{k\rightarrow \infty}x_k=a$.}\\
	\underline{Bew.:} \underline{(a)} \oraul{Eindeutigkeit: Ann.:} u,v seien 
	Fixpunkte, $u\neq v$ \\
	Dann:$0\neq ||u-v|| = ||f(u)-f(v)||\leq p||u-v||\textless ||u-v||, 
	\lightning$. Also folgt u=v.\\
	\oraul{Existenz: 1. Ableitung:} $||x_{k+1}-x_k||=||f(x_k)-f(x_{k-1})||\leq 
	p||x_k-x_{k-1}||\leq\cdots\leq p^k||x_1-x_0||.$\\
	\oraul{2. Ableitung:} $||x_{k+l}-x_k||\leq 
	(p^{k+l-1}+p^{k+l-2}+\cdots+p^k)||x_1-x_0||$ (mit $l\geq 1$)\\
	=$p^k(p^{l-1}+\cdots+1)||x_1-x_0||\leq 
	\frac{p^k}{1-p}||x_1-x_0||.\orangecircle{*}$\\
	Es folgt: $(x_k)$ ist eine \oraul{Cauchyfolge}, also ex. 
	\oraul{$\lim\limits_{k\rightarrow \infty}x_k=a \in U$} (Vgl. \redcircle{Ü} 
	\blueul{Bl. 2, A1.2.}: Kgz. in $\mathbb{R}^n \checkmark$\\
	Dann :Kgz. in U, da U beschr. (in $||\cdot||_\infty$) und abgeschlossen.)\\
	Dieser GW ist Fixpunkt, denn $f(a)\xleftarrow{k\rightarrow\infty\text{f 
	stetig}}f(x_k)=x_{k+1}\xrightarrow{k\rightarrow\infty}a$,\\
	und aufgrund der Eindeutigkeit des GWes folgt f(a)=a.\\
	(b): Obige Abschätzung \orangecircle{*} für $ l\rightarrow\infty$.\\
	\strut\hfill$\square$\\
	\textbf{8.7. \underline{Bsp.:}} \redul{Newton-Verfahren} zur numerischen 
	\redul{Nullstellenbestimmung:}\\
	Für $f:\mathbb{R}\rightarrow\mathbb{R}$ diff'bar und $\forall x:f'(x)\neq 
	0\\$ setze \redul{$g(x):=x-\frac{f(x)}{f'(x)}$}, 
	$g:\mathbb{R}\rightarrow\mathbb{R}.$ ein Fixpunkt a von g ist dann genau 
	eine Nst. von f wegen $g(a)=a\Leftrightarrow 
	a=a-\frac{f(a)}{f'(a)}\Leftrightarrow f(a)=0$.\\
	Die Fixpunkte von g bzw. Nst. von f erhält man nach \blueul{8.6. (b)} mit 
	der Rekursion \greenul{$x_0\in\mathbb{R}, 
	x_{k+1}:=g(x_k)=x_k-\frac{f(x_k)}{f'(x_k)}$}, \underline{falls g 
	Kontrahierend ist.} Ist a eine Nst. von f in einen IV I, $f\in l^2(I)$, so 
	ist laut \blueul{Taylorformel (1. Ordnung)}: 
	$0=f(a)=f(x)+f'(x)(a-x)+\frac{f''(\xi)}{2}(a-x)^2, \xi$ zw. a und x\\
	$\Leftrightarrow 
	\underbrace{x-\frac{f(x)}{f'(x)}}_{g(x)}-a=\frac{f''(\xi)}{2f'(x)}(x-a)^2$, 
	mit $M_2=sup_{x\in I}|f''(x)|, m_1=inf_{x\in I}|f'(x)|$\\
	folgt $|g(x)-a|\leq \frac{M_2}{2m_1}|x-a|^2$ für alle $x \in I,$\\
	was unter bestimmten Voraussetzungen zum Nachweis der 
	Kontraktionseigenschaft von g führt.\\
	\\
	\textbf{8.8. \redul{Satz über die lokale Umkehrbarkeit:}} (Verallg. von 
	\blueul{Satz An12.2, Ableitung von Umkehrfktn.})\\
	\underline{Vor.:} \greenul{$f\in l^1(D,\mathbb{R}^n, a \in D\subset 
	\mathbb{R}^n), A:=f'(a)\in\mathbb{R}^{nxm}$} sei \greenul{invertierbar, 
	b:=f(a)}.\\
	\underline{Beh.:} (1) \greenul{$\exists U \subset \mathbb{R}^n \exists V 
	\subset \mathbb{R}^n$} : $a\in U\xrightarrow{f} V \ni b$ 
	\greenul{bijektiv},\\
	(2) \greenul{$f^{-1}_{ru}$} ist \greenul{stetig diff'bar} (in V),\\
	(3) \greenul{$f'(x)$} $\in \mathbb{R}^{nxm}$ ist \greenul{invertierbar für 
	alle $x\in U$},\\
	(4) \greenul{$\forall y \in V: 
	(f^{-1}_{ru})'(y)=(f'(f^{-1}_{ru}(y)))^{-1}$}. \textopencorner Vgl. 
	$(f^{-1})'(b)=(f'(f^{-1}(b)))^{-1}$ in \blueul{An 12.2}\textcorner\\
	\underline{Bew.:} $\bullet$ Sei $\OE$ \oraul{a=b=o}, durch Betrachtung von 
	f(x-a) bzw. f(x)-f(a). Ferner sei \oraul{$\OE A = I_n$}, die Einheitsmatrix 
	$I_n=\begin{pmatrix}
		1&\cdots&0\\
		0&1&0\\
		0&\cdots&1
	\end{pmatrix}\in \mathbb{R}^{nxm}$, durch Betrachtung von 
	\oraul{$x\rightarrowtail A^{-1}\circ(f(x+a)-b)$} mit der Ableitung 
	$A^{-1}\circ (f(x+a)-b)'(a)\underbrace{=}_{a=b=o} A^{-1}\circ 
	f'(o)=A^{-1}\circ A=T_n.$\\
	$\bullet$ \underline{Ann.:} \blueul{(1) und (2)} gelte. Für $x\in U$ gilt 
	dann: $f^{-1}_{ru}f(x)=x$, was laut \blueul{Kettenregel} diffbar ist mit 
	der Ableitung: $(f^{-1}_{ru})'(f(x))$ \underline{f'(x)}=$I_n$, denn 
	$f^{-1}_{ru}$ ist diff'bar \blueul{laut (2)}. Also ist $f'(x)$ invertierbar 
	(also (3)), nämlich mit $(f'(x))^{-1}=(f^{-1}_{ru})'(f(x))$. Setze nun 
	$y=f_{ru}(x)=f(x)$, also x= $\textcolor{magenta}{f^{-1}_{ru}(y)}$, es folgt 
	$(f'(\textcolor{magenta}{f^{-1}_{ru}(y)}))^{-1}=(f^{-1}_{ru})'(y)$, das ist 
	Formel (4).\\
	\\
	$\bullet$ \underline{Noch z.z.:} \blueul{(1) und (2)}. Dazu betr. die Norm 
	$||\cdot||_\infty$. Nach Vor. ist f' stetig (d.h. die $f_i'$ stetig), sowie 
	$f'(0)=I_n$. Daher $\exists s \textgreater0 \forall x \in U^{2s}_0: 
	max_i||f_i'(x)^T-pr_i(I_n)||_\infty\leq \frac{1}{n}=:M$, dabei sei $\OE 
	\overline{U^{2s}_0}\subseteq D.$ (Def. 
	$\overline{U^{2s}_0}:=\{x\in\mathbb{R}^n;||x||_\infty\leq2s\}.$)\\
	Setze \oraul{V:=$U_0^s$}, wähle \oraul{$y\in V$ fest} und setze als 
	Hilfsfunktion \oraul{$h(x):=x-f(x)+y$} für $x\in D$.\\
	Dann ist $h'(x)=I_n-f'(x)$, sowie 
	$\max_{i}||pr_i(I_n)-f_i'(x)^T||\infty\leq \frac{1}{2n}=M$,\\
	es folgt mit \blueul{Flogerung 6.6 des MWS},\\
	dass $||h(x_1)-h(x_2)||_\infty \leq n M \cdot 
	||x_1-x_2||_\infty=\frac{1}{2}||x_1-x_2||$ für alle $x_1.x_2 \in 
	\overline{U^{2s}_0}$,\\
	d.h. h ist \oraul{Kontrahierend}. Nun ist \oraul{$\overline{U^{2s}_0}$ 
	beschränkt und abgeschlossen}, daher wende den \blueul{Banachschen 
	Fixpunktsatz 8.6.} für h an. Dies zeigt:\\
	$\exists ! x \in \overline{U^{2s}_0}$ mit $ h(x)=x\Leftrightarrow 
	x-f(x)+y=x \Leftrightarrow f(x)=y=h(0).$\\
	Problem: Liegt x auf dem Rand von $\overline{U^{2s}_0}$ ? Nein: 
	$h(\overline{x})$\\
	\textopencorner Haben die Abschätzung 
	$||h(\overline{h})||_\infty=||\underbrace{(\overline{x}-f(\overline{x})+y)}_{h(\overline{x})}
	 -\underbrace{h(0)}_{=y}+\underbrace{h(0)}_{=y}||_\infty\\
	\leq \frac{1}{2}\underbrace{||\overline{x}-0||_\infty}_{\leq2s}+ 
	\underbrace{||y||_\infty}_{\textless 2s}\textless 2s$ für alle 
	$\overline{x} \in \overline{U^{2s}_0}$,\\
	mit h(x)=x folgt $||x||_\infty \leq \frac{1}{2}||x||_\infty+||y||_\infty$ 
	bzw. \oraul{$||x||_\infty \leq 2||y||_\infty\textless 2s$}\blackcircle{*}, 
	d.h. \oraul{$x\in \overline{U^{2s}_0}$}.\textcorner\\
	Setze \oraul{$U:=f^{-1}(U_0^s)\cap U^{2s}_0$}$\subset \mathbb{R}^n, 
	\Uppsi:= f_{ru}, U\xrightarrow{\Uppsi} V$ ist alos injektiv und surjektiv, 
	also bijektiv $\rightarrow(1)$ gilt. Setze $l:=\Uppsi^{-1}=f^{-1}_{ru}.$\\
	Ans \blackcircle{*} folgt: l ist stetig in o, 
	\textopencorner$||l(y)-l(o)||_\infty\leq 2||y-o||_\infty\textcorner$\\
	$\bullet$  \oraul{Beh.:} l ist in o diff'bar und \oraul{$l'(o)=I_n$}, d.h. 
	(2) gilt.\\
	\oraul{Bew.:} $y=f(x)=\underbrace{o}_{f(o)}+x+\epsilon$ ist in o stetig,\\
	$\epsilon (x)\xrightarrow{x\rightarrow o}o$ da $f'(o)=I_n$.\\
	$\Rightarrow 
	l(y)=x=y-\epsilon(l(y))\cdot\frac{||l(y)||_\infty\cdot||y||_\infty}{||y||_\infty}.$\\
	\\
	Ans $y\rightarrow o$ folgt $l(y)\rightarrow o$, da l stetig in o, ebenso 
	gilt $\epsilon (l(y))\xrightarrow{y\rightarrow o}o.$ Also: 
	$\epsilon(l(y))\cdot\frac{||l(y)||_\infty}{||y||_\infty}\xrightarrow{y\rightarrow
	 o}0.$\\
	Daher ist l in o diff'bar und $l'(o)=I_n. \checkmark$\\
	$\bullet$ Ferner ist $l'=(f^{-1}_{ru})'$ stetig als \oraul{Komposition 
	stetiger} Abbildungen nach (4). (bemerke, dass im obigen Beweis von (4) 
	nicht die Stetigkeit von l' benutzt wird.)\\
	\strut\hfill$\square$\\
	\textbf{8.9. \underline{Zusatz:}} Es gilt auch: \greenul{$f:_{ru}\in 
	l^k(U,\mathbb{R}^n)\Rightarrow l=(f_{ru})^{-1}\in l^k(U,\mathbb{R}^n)$}.\\
	\\
	\textbf{8.10. \underline{Bem.:}} Eine Abb. $f:U\rightarrow V$ mit 
	$U,V\subset \mathbb{R}^n$ heißt \redul{Diffeomorphismus}, falls f $b_{ij}$ 
	und $f,f^{-1}$ stetig db.
	
	
	
	
	
	
	
	
	
	
	
	
	
	
	
	
	
	
	
	
	
	
	
	
\end{document}